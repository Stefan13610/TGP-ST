\documentclass[11pt,a4paper]{article} 
\usepackage[utf8]{inputenc} 
\usepackage[T1]{fontenc}   
\usepackage{amsmath}        
\usepackage{amsfonts}       
\usepackage{amssymb}       
\usepackage{graphicx}       
\usepackage[english]{babel} 
\usepackage{hyperref}      
\usepackage{authblk}       


\newcommand{\PhiZero}{\Phi_0}

\begin{document}

\title{Towards the Unification of Gravity and Quantum Mechanics}
\begin{center}
\textit{Disclaimer: The original version of this manuscript was written in Polish. This English translation was generated automatically and may contain minor linguistic inaccuracies. The abbreviation \textbf{TGP} stands for \textit{Teoria Generowanej Przestrzeni} (Theory of Generated Space). The suffix \textbf{ST} reflects the fact that the work is still at a very early stage of development. The author welcomes feedback and corrections.}
\end{center}

\author{Mateusz Serafin}
\affil{\small Polska, Kraków} 

\date{\today}

\maketitle 


\section{Abstract}
\label{sec:Abstract}

This paper presents an outline of the \textbf{Generated Space Theory (TGP-ST)}, a novel approach to the unification of gravity and quantum mechanics. The fundamental assumption of TGP-ST is that \textbf{space is not a passive background, but an emergent, dynamic entity directly generated by the presence of mass and energy}. This leads to the rejection of the traditional concept of vacuum as empty space; instead, the Universe is viewed as a network of dynamically generated space, whose “tension” is described by a scalar field $\Phi$. In TGP-ST, even what we perceive as “vacuum” is an active state of quantum fluctuations of the $\Phi$ field, originating from space-generating particles. In this context, \textbf{virtual particles are interpreted as local manifestations of quantum fluctuations of this generated space}, arising from the uncertainty in the position of the space-generating sources themselves.

Within the framework of TGP-ST, fundamental physical constants such as the speed of light $c$, Planck’s constant $\hbar$, and the gravitational constant $G$ cease to be universal constants and instead become local fields dependent on $\Phi$. We show how these dependencies preserve the constancy of the Planck length $\ell_P$. The theory offers potential resolutions to long-standing problems in physics, including the elimination of singularities in black holes, a reinterpretation of dark energy, and an explanation of the Hubble tension. We present modified equations of motion for gravity and the $\Phi$ field, as well as the TGP-Schrödinger equation with an emergent Planck constant. We also discuss implications for the classical limit and propose testable observational predictions that could verify the dynamic nature of spacetime and fundamental constants.

\end{abstract}


\vspace{1cm} 


\noindent\textbf{Keywords:} Generated Space Theory; Quantum Gravity; Unification; Variable Speed of Light; Emergent Space; Fundamental Constants.

\newpage

\section{Introduction}
\label{sec:Introduction}

Modern physics faces the fundamental challenge of unifying two of its most successful yet incompatible frameworks: Einstein's General Theory of Relativity (GR) and Quantum Mechanics (QM). GR successfully describes gravity as a geometric property of spacetime, explaining phenomena ranging from planetary motion to the dynamics of entire galaxies. In contrast, QM, which forms the foundation of the Standard Model of particle physics, describes nuclear and electromagnetic interactions in the microscopic world with remarkable precision. Unfortunately, these two theories are incompatible in extreme conditions, such as the interiors of black holes or the early Universe, where gravitational and quantum effects become equally significant.

This incompatibility gives rise to a number of unresolved problems: the appearance of singularities in the equations of GR (e.g., at the centers of black holes, where density becomes infinite), the cosmological constant problem (the discrepancy between dark energy and QM predictions), and the general absence of a quantum theory of gravity. Efforts to develop a "Theory of Everything" that unifies all fundamental forces have been ongoing for decades, producing many promising but still incomplete approaches, such as string theory and loop quantum gravity.

This paper outlines the \textbf{Generated Space Theory (TGP-ST)}, a novel approach that proposes a fundamental paradigm shift in our understanding of spacetime and gravity. Unlike GR, where mass and energy curve a preexisting spacetime, TGP-ST postulates that \textbf{space is not a passive backdrop but a dynamic emergent entity, generated directly by the presence of mass and energy}. This means that the existence of space is inseparably linked to the presence of matter, and the traditional concept of “vacuum” as empty space ceases to be applicable. Instead, what we perceive as vacuum is a dynamic state of quantum fluctuations of the space-generating field, sourced by existing particles. In this context, virtual particles are not phenomena emerging from nothingness but local manifestations of fluctuations in the generated space, resulting from the uncertainty in the position of the space-generating sources.

Within TGP-ST, we introduce a scalar field $\Phi$ that intuitively describes the “density” or “tension” of generated space. This dynamic field plays a key role not only in describing gravity but also in determining the dynamic nature of fundamental physical constants. We demonstrate how the speed of light ($c$), Planck’s constant ($\hbar$), and the gravitational constant ($G$) cease to be universal constants and instead become local fields dependent on $\Phi$, while still preserving the constancy of the fundamental Planck scale ($\ell_P$).

The aim of this article is to present a coherent formalism of TGP-ST, to demonstrate its potential for offering new perspectives on unresolved problems in physics, and to identify testable predictions. The structure of the paper is as follows: Section~\ref{sec:Zalozenia} introduces the core assumptions of TGP-ST. Section~\ref{sec:FormulacjaKlasyczna} elaborates on the classical formulation of the theory, including the fundamental action and equations of motion. In Section~\ref{sec:StaleDynamiczne}, we derive the dynamic dependence of fundamental constants on the field $\Phi$. Section~\ref{sec:MechanikaKwantowa} extends TGP-ST into the quantum domain, introducing a modified Schrödinger equation and a reinterpretation of the nature of virtual particles. Section~\ref{sec:GranicaKlasyczna} discusses two distinct paths to the classical limit. In Section~\ref{sec:Konsekwencje}, we examine key physical implications and how TGP-ST potentially resolves existing problems. Section~\ref{sec:Testy} presents testable predictions, and Section~\ref{sec:OtwarteProblemy} discusses open questions and future research directions.

---

\section{Fundamental Assumptions of the Generated Space Theory (TGP-ST)}
\label{sec:Zalozenia}

The Generated Space Theory (TGP-ST) is based on a radical shift in our fundamental understanding of the nature of spacetime. In contrast to the standard model of physics, where spacetime is viewed as a passive, preexisting backdrop on which physical phenomena occur, TGP-ST proposes a different perspective. The central, overarching assumption of the theory is that \textbf{space is not an independent entity, but an emergent, dynamic construct that is generated directly by the presence of mass and energy}.

This fundamental assumption has several key implications:

\begin{enumerate}
    \item \textbf{Space as a Dynamic Network:} Every particle, every field, and every form of energy in the Universe actively contributes to the \textbf{creation and shaping of space itself}. This represents a profound rejection of the concept of absolute, empty space. Instead, space is conceived as a kind of dynamic “network” or “fabric” whose existence and properties are inseparably linked to the material content it contains. Without mass and energy, there is no space.

    \item \textbf{Scalar Field $\Phi$ as Spatial Density:} In TGP-ST, we introduce the \textbf{scalar field $\Phi$} as the fundamental carrier of emergent space. The field $\Phi$ can be intuitively understood as a measure of the “density,” “tension,” or “concentration” of locally generated space. Its value at a given point in spacetime reflects the intensity of space generation at that location. The vacuum value $\Phi_0$ serves as a reference point for this “density.”

    \item \textbf{The Disappearance of the Traditional Vacuum Concept and the Nature of Virtual Particles:}
    A key consequence of TGP-ST is the \textbf{elimination of the traditional concept of vacuum as empty space}. In TGP-ST, there is no “nothingness” in the sense of a complete absence of being. Even in regions devoid of classical matter or fields, there must exist a non-zero, fluctuating $\Phi$ field, which constitutes the \textbf{minimum level of generated space}. This “vacuum” field $\Phi_0$ (or more generally, the ground state of $\Phi$) is a fundamental component of the very structure of spacetime.

    Furthermore, TGP-ST introduces a new interpretation of \textbf{virtual particles}. These are not ephemeral entities that appear and vanish from nothingness. Instead, they are \textbf{local, temporary manifestations of quantum fluctuations of the generated space itself}. Although a particle may classically have a well-defined position, it quantum-mechanically generates space that is subject to fluctuations. These fluctuations cause a “blurring” of the generated space’s localization. Virtual particles are precisely these “waves” or “distortions” of the generated space, which can be detected beyond the immediate, classical location of the source particle. Thus, what we perceive as vacuum is a dynamic state of ongoing quantum fluctuations of the $\Phi$ field.

    \item \textbf{Gravity as an Effect of Space Generation:}
    This new approach to the nature of space leads to profound modifications in our understanding of gravity. Gravity is no longer simply the curvature of a preexisting geometry, but \textbf{a direct result of the dynamic creation and interaction of that geometry by matter itself}. The more matter (or energy) present, the more space is generated, and the properties of this generated space (described by the field $\Phi$) determine the observed gravitational interactions.
\end{enumerate}


---

\section{Classical Formulation of TGP-ST}
\label{sec:FormulacjaKlasyczna}

The classical dynamics of the Generated Space Theory (TGP-ST) are formulated using the principle of least action. The action encompasses the dynamics of spacetime (through the metric tensor $g_{\mu\nu}$) and the scalar field $\Phi$, which describes the “density” of generated space. The total action $S$ consists of a gravitational term, a kinetic and potential term for the field $\Phi$, a coupling term between $\Phi$ and matter, and the matter Lagrangian:

\begin{equation}
    S = \int d^4x \sqrt{-g} \left[ \frac{1}{16\pi G_0}R + \frac{1}{2}\Phi R + \frac{1}{2}g^{\mu\nu}(\partial_\mu\Phi)(\partial_\nu\Phi) - V(\Phi) - g_{c}\Phi T^\mu_\mu + L_{matter} \right]
    \label{eq:FundamentalAction}
\end{equation}

The individual components of the action can be interpreted as follows:

\begin{enumerate}
    \item \textbf{Einstein-Hilbert Term:} $\frac{1}{16\pi G_0}R$  
    This is the standard gravitational action term in General Relativity, where $R$ is the Ricci scalar and $G_0$ is the “base” gravitational constant (corresponding to the vacuum value for $\Phi_0$). This term describes the fundamental geometry of spacetime.

    \item \textbf{Non-Minimal Coupling Term:} $\frac{1}{2}\Phi R$  
    This term represents a non-minimal coupling of the scalar field $\Phi$ to the curvature of spacetime. It plays a key role in scalar-tensor theories where the scalar field actively contributes to gravitational dynamics. In TGP-ST, it formalizes the intuition that the “density” of generated space ($\Phi$) directly influences the geometry (represented by $R$).

    \item \textbf{Kinetic Term for the Scalar Field:} $\frac{1}{2}g^{\mu\nu}(\partial_\mu\Phi)(\partial_\nu\Phi)$  
    This term describes the dynamics and propagation of the scalar field $\Phi$. It is analogous to the kinetic term for standard scalar fields in quantum field theory.

    \item \textbf{Self-Interaction Potential for $\Phi$:} $V(\Phi)$  
    The potential $V(\Phi)$ governs the dynamics of $\Phi$ in vacuum (i.e., in the state of minimally generated space) and determines its stability and self-interactions. The proposed form of the potential is:
    \begin{equation}
        V(\Phi) = -\frac{|\lambda|}{4}\Phi^3 + \frac{\kappa}{8}\Phi^4 + V_0
        \label{eq:ScalarPotential}
    \end{equation}
    where $\lambda$ and $\kappa$ are coupling constants, and $V_0$ is an offset constant. Stability conditions require $\lambda < 0$ and $\kappa > 0$. This potential is crucial for generating an effective cosmological constant and for screening mechanisms involving the field $\Phi$.

    \item \textbf{Coupling to Matter Term:} $-g_{c}\Phi T^\mu_\mu$  
    This term describes a direct coupling of the scalar field $\Phi$ to the trace of the energy-momentum tensor of matter $T^\mu_\mu$. The constant $g_{c}$ represents the strength of this coupling. In TGP-ST, this term formalizes the idea that matter is the source of the $\Phi$ field, which in turn generates the space in which matter exists and interacts.

    \item \textbf{Matter Lagrangian:} $L_{matter}$  
    This represents the Lagrangian of all other matter fields, such as those in the Standard Model, describing their dynamics and interactions independently of the $\Phi$ field.
\end{enumerate}


\subsection{Equations of Motion in TGP-ST}
The equations of motion of the theory are derived by varying the action \eqref{eq:FundamentalAction} with respect to the independent fields: the metric tensor $g_{\mu\nu}$ and the scalar field $\Phi$.

\subsubsection{Einstein Equation (variation with respect to $g_{\mu\nu}$)}
Varying the action with respect to $g_{\mu\nu}$ yields the modified gravitational field equations:
\begin{equation}
    \left(\frac{1}{16\pi G_0} + \frac{\Phi}{2}\right) G_{\mu\nu} = \frac{1}{2}T^{\Phi}_{\mu\nu} + \frac{1}{2}T^{\text{matter}}_{\mu\nu} + \nabla_\mu\nabla_\nu\Phi - g_{\mu\nu}\square\Phi
    \label{eq:EinsteinTGP}
\end{equation}
where $G_{\mu\nu}$ is the Einstein tensor, and $T^{\Phi}_{\mu\nu}$ is the energy-momentum tensor of the scalar field $\Phi$, defined as:
\begin{equation}
    T^{\Phi}_{\mu\nu} = \partial_\mu\Phi \partial_\nu\Phi - \frac{1}{2}g_{\mu\nu}(\partial\Phi)^2 - g_{\mu\nu}V(\Phi)
\end{equation}
In this equation, the effective gravitational constant becomes dependent on the field $\Phi$, which is a characteristic feature of scalar-tensor theories.

\subsubsection{Equation for the Scalar Field $\Phi$ (variation with respect to $\Phi$)}
Varying the action with respect to the scalar field $\Phi$ yields the equation describing the dynamics of space generation:
\begin{equation}
    \square\Phi + \frac{1}{2}R + \frac{dV}{d\Phi} = g_{c} T^\mu_\mu
    \label{eq:ScalarFieldEquation}
\end{equation}
This equation describes the propagation of the field $\Phi$, its coupling to the curvature of spacetime ($R$), and its interaction with matter (via the trace of the energy-momentum tensor $T^\mu_\mu$). The term $\frac{dV}{d\Phi}$ acts as an effective mass term for the $\Phi$ field and plays a role in screening mechanisms.

\subsection{Interpretation of the Field $\Phi$ in the Classical Formulation}
In the classical picture, the field $\Phi$ is a direct carrier of gravity, modulating its strength and range. Solutions for $\Phi$ in the presence of mass sources (such as stars or galaxies) determine the local geometry of spacetime and the properties of the generated space. This classical perspective provides the foundation for understanding how TGP-ST differs from GR on macroscopic scales and potentially explains phenomena such as dark energy and cosmological discrepancies.

---

\section{Quantum Mechanics in TGP-ST}
\label{sec:MechanikaKwantowa}

In the Generated Space Theory (TGP-ST), quantum mechanics is deeply intertwined with the dynamic nature of spacetime. Standard fundamental constants, such as Planck's constant $\hbar$, become variable fields that depend on the local “density” of generated space, described by the scalar field $\Phi$. This leads to a modified Schrödinger equation that reflects this fundamental dependency.

\subsection{TGP-Schrödinger Equation}
The Schrödinger equation in TGP-ST for a single particle (or for the center of mass of a particle system, in the context of emergent $\hbar_{eff}$) takes the form:
\begin{equation}
    i\hbar_{eff}(x,t)\frac{\partial\Psi(x,t)}{\partial t} = \left[-\frac{\hbar_{eff}(x,t)^2}{2m}\nabla^2 + V_{ext}(x) + \beta\Phi(x,t) + \lambda\left(\frac{\nabla^2\Phi(x,t)}{\Phi(x,t)}\right)\right]\Psi(x,t)
    \label{eq:SchrodingerTGP}
\end{equation}
where $\Psi(x,t)$ is the particle’s wave function, $m$ is its mass, and $V_{ext}(x)$ is an external potential. Each term reflects the specific structure of TGP-ST:

\subsubsection{Emergent Planck Constant ($\hbar_{eff}$)}
In TGP-ST, Planck's constant is not a fundamental universal constant, but rather an \textbf{emergent quantity} that depends on the local “density” of generated space and the number of particles in the system. Its dependence on the field $\Phi$ and particle count $N$ (as discussed in Section~\ref{sec:StaleDynamiczne}) is given by:
\begin{equation}
    \hbar_{eff}(x,t) = \hbar_0 \left(\frac{\Phi_0}{\Phi(x,t)}\right)^{1/2}
    \label{eq:h_eff}
\end{equation}
where $\hbar_0$ is the “baseline” Planck constant, and $\Phi_0$ is the value of the field $\Phi$ in vacuum. The value of $N$ is accounted for globally by scaling $\hbar_0$ for a given system. In macroscopic contexts where $N$ is large, this affects the quantum scale of the system, leading to the suppression of relative quantum effects (see Section~\ref{sec:GranicaKlasyczna}).

\subsubsection{Nonlocal Coupling to Space}
The term $\beta\Phi(x,t)$ represents the coupling of the wave function to the field $\Phi$, which is dynamically generated by matter. The suggested form of this coupling is a nonlocal potential:
\begin{equation}
    \beta\Phi(x,t) = g_s\int \frac{|\Psi(x',t)|^2}{|x-x'|} \, d^3x'
    \label{eq:NonlocalCoupling}
\end{equation}
where $g_s$ is a coupling constant (related to $G_0$ and $c_0$). This term introduces \textbf{self-consistency} between the wave function and the geometry/density of space: the probability density $|\Psi|^2$ of the particle influences the field $\Phi$, which in turn affects the particle’s dynamics. This is a novel type of interaction not present in standard quantum mechanics.

\subsubsection{Geometric-Quantum Term}
The term $\lambda\left(\frac{\nabla^2\Phi}{\Phi}\right)$ directly links the wave function to the local geometry of the generated space. It can be interpreted as the influence of the local curvature or “tension” of the field $\Phi$ on quantum dynamics. In some limits, it may be related to the Ricci scalar $R$ from the TGP-ST gravitational field equations.

\subsection{Nature of Virtual Particles and Vacuum Concept}
In TGP-ST, the traditional concept of vacuum as empty space is replaced by a \textbf{dynamic network of generated space}. Every particle that possesses energy or mass is a source of generated space. This space, though sourced by a classically defined particle, itself undergoes quantum fluctuations.

\textbf{Virtual particles} in TGP-ST are interpreted as \textbf{local, temporary manifestations of quantum fluctuations of the generated space}. They are not particles arising from nothingness in the sense of the standard energy-time uncertainty principle, but rather “echoes” or “waves” of generated space that propagate outward from source particles. These fluctuations arise from quantum \textbf{uncertainty in the localization of the generated space itself}, which translates into positional uncertainty in the classical sense. Thus, what we perceive as “vacuum” is in fact a dynamic, continuously fluctuating state of the $\Phi$ field, filled with these manifestations of generated space, which in turn influence fundamental properties such as the speed of light $c(\Phi)$.

\subsection{Key Implications of Quantum Mechanics in TGP-ST}
The modified Schrödinger equation and reinterpretation of fundamental constants lead to deep consequences:
\begin{itemize}
    \item Quantum effects depend on local spatial conditions (i.e., the value of $\Phi$).
    \item Quantum interactions are inseparably linked with the dynamics of space itself.
    \item In extreme conditions (e.g., near massive objects) where $\Phi$ is high, $\hbar_{eff}$ approaches zero, implying a transition to classical behavior even for individual particles.
\end{itemize}

---

\section{Dynamic Fundamental Constants in TGP-ST}
\label{sec:StaleDynamiczne}

In the standard framework of the Standard Model of physics and General Relativity, fundamental constants such as the speed of light $c$, Planck’s constant $\hbar$, and the gravitational constant $G$ are treated as immutable universal values. In the Theory of Generated Space (TGP-ST), with its foundational assumption of the emergent and dynamic nature of space, these “constants” are no longer fundamental in the traditional sense. Instead, they become \textbf{dynamic fields that depend on the local “density” of generated space, described by the scalar field $\Phi$}. This dependency is a key element of the theory, enabling the unification of quantum and gravitational phenomena.

For simplicity, we assume that the value of the field $\Phi$ in the vacuum state (minimally generated space) is $\Phi_0$. We normalize this to unity ($\Phi_0 = 1$) in subsequent calculations. The values $c_0$, $\hbar_0$, and $G_0$ are the \textbf{observed values of these constants in our local relativistic vacuum (i.e., when $\Phi = \Phi_0$)}, which TGP-ST treats as a reference point for their dynamic variability.

\subsection{Speed of Light ($c(\Phi)$)}
In TGP-ST, the speed of light $c$ is not a universal constant but a dynamically varying quantity dependent on the local “density” of generated space $\Phi$. Intuitively, the “denser” the space (i.e., the larger the value of $\Phi$), the more difficult it is for electromagnetic waves to propagate, resulting in a lower speed of light.

To introduce this dependency formally, we modify the electromagnetic action by introducing a coupling function $Z(\Phi)$ with the electromagnetic field tensor $F_{\mu\nu}$:
\begin{equation}
    S_{EM} = \int d^4x \sqrt{-g} \left[ -\frac{1}{4} Z(\Phi) F_{\mu\nu} F^{\mu\nu} \right]
    \label{eq:ActionEM}
\end{equation}
We propose the following form of the coupling function, which directly links the properties of space to the propagation of light:
\begin{equation}
    Z(\Phi) = \frac{\Phi}{\Phi_0}
    \label{eq:ZPhi}
\end{equation}
Varying the action \eqref{eq:ActionEM} with respect to the electromagnetic potential $A_\mu$ leads to modified Maxwell equations. Analyzing the phase velocity of electromagnetic waves in this medium (where $Z(\Phi)$ acts as a variable dielectric coefficient) yields the following relation for the speed of light as a function of the field $\Phi$:
\begin{equation}
    c(\Phi) = c_0 \left(\frac{\Phi_0}{\Phi}\right)^{1/2} = \frac{c_0}{\sqrt{\tilde{\Phi}}}
    \label{eq:cPhi}
\end{equation}
where $c_0$ is the speed of light in vacuum (for $\Phi = \Phi_0$, i.e., $\tilde{\Phi} = 1$). This expression shows that in regions of high $\Phi$ (densely generated space), $c(\Phi)$ decreases, approaching zero as $\Phi \rightarrow \infty$.

\subsection{Planck’s Constant ($\hbar(\Phi)$)}
In TGP-ST, Planck’s constant $\hbar$ is also a dynamic field, which is fundamentally important for unifying quantum mechanics with gravity. As discussed in Section~\ref{sec:MechanikaKwantowa}, the effective Planck constant $\hbar_{\text{eff}}$ in the TGP-Schrödinger equation depends on the local value of the field $\Phi$. The proposed dependence mirrors that of the speed of light, which is intuitively consistent since both constants characterize propagation in spacetime:
\begin{equation}
    \hbar(\Phi) = \hbar_0 \left(\frac{\Phi_0}{\Phi}\right)^{1/2} = \frac{\hbar_0}{\sqrt{\tilde{\Phi}}}
    \label{eq:hbarPhi}
\end{equation}
where $\hbar_0$ is the vacuum value of Planck’s constant. This form ensures that in regions of extremely high $\Phi$ (e.g., near central masses of black holes), $\hbar(\Phi)$ tends to zero. This is a key mechanism for the suppression of quantum effects in strong fields, leading to classical behavior of matter in those regions (see Section~\ref{sec:GranicaKlasyczna}).

\subsection{Gravitational Constant ($G(\Phi)$)}
To maintain the fundamental consistency between quantum gravity and dynamic constants, we postulate that the Planck length $\ell_P = \sqrt{\hbar G / c^3}$ remains a universal constant. If $\ell_P$ is to be constant while $c$ and $\hbar$ are dynamic, then the gravitational constant $G$ must also be a dynamic field $G(\Phi)$.

We derive the dependence $G(\Phi)$ from the condition of fixed $\ell_P^2 = \hbar G / c^3$:
\[
    G(\Phi) = \frac{\ell_P^2 c(\Phi)^3}{\hbar(\Phi)}
\]
Substituting the previously defined dependencies for $c(\Phi)$ and $\hbar(\Phi)$:
\[
    G(\Phi) = \frac{\ell_P^2 \left(c_0 \left(\frac{\Phi_0}{\Phi}\right)^{1/2}\right)^3}{\hbar_0 \left(\frac{\Phi_0}{\Phi}\right)^{1/2}} 
    = \frac{\ell_P^2 c_0^3 (\Phi_0/\Phi)^{3/2}}{\hbar_0 (\Phi_0/\Phi)^{1/2}}
\]
Since $\ell_P^2 = \hbar_0 G_0 / c_0^3$, we substitute:
\[
    G(\Phi) = \frac{\hbar_0 G_0}{c_0^3} \cdot \frac{c_0^3 (\Phi_0/\Phi)^{3/2}}{\hbar_0 (\Phi_0/\Phi)^{1/2}} = G_0 \left(\frac{\Phi_0}{\Phi}\right) = \frac{G_0}{\tilde{\Phi}}
\]
Thus, the gravitational constant $G$ is also a dynamic field:
\begin{equation}
    G(\Phi) = G_0 \frac{\Phi_0}{\Phi} = \frac{G_0}{\tilde{\Phi}}
    \label{eq:GPhi}
\end{equation}
In this case, $G(\Phi)$ decreases as $\Phi$ increases.

\subsection{Consistency of the Planck Scale}
Applying the above dependencies $c(\Phi)$, $\hbar(\Phi)$, and $G(\Phi)$ to the definition of the Planck length $\ell_P$ confirms its constancy:
\[
    \ell_P^2 = \frac{\hbar(\Phi) G(\Phi)}{c(\Phi)^3} = \frac{\hbar_0 (\Phi_0/\Phi)^{1/2} \cdot G_0 (\Phi_0/\Phi)^{1/2}}{(c_0 (\Phi_0/\Phi)^{1/2})^3}
\]
\[
    = \frac{\hbar_0 G_0}{c_0^3} \left(\frac{\Phi_0}{\Phi}\right)^{1/2 + 1/2 - 3/2} = \frac{\hbar_0 G_0}{c_0^3} \left(\frac{\Phi_0}{\Phi}\right)^0 = \frac{\hbar_0 G_0}{c_0^3} = \text{constant}
\]
This constancy of the Planck length $\ell_P$ is a key result confirming the internal consistency of TGP-ST. Despite the dynamic nature of individual fundamental constants, their interdependence ensures that the fundamental scale of quantum gravity remains unchanged. This suggests that TGP-ST provides a coherent framework in which variability of the constants emerges naturally from the emergent nature of spacetime.

---

\section{Classical Limit in TGP-ST}
\label{sec:GranicaKlasyczna}

In standard quantum mechanics, the classical limit is traditionally approached as Planck’s constant $\hbar$ tends to zero. In the Generated Space Theory (TGP-ST), due to the dynamic nature of $\hbar(\Phi)$, there exist \textbf{two distinct but complementary pathways to reach the classical limit}, depending on the physical context. This dual approach is a key novelty of TGP-ST, enabling a coherent transition from the quantum realm to the macroscopic world.

\subsection{Statistical Limit ($N \rightarrow \infty$)}
The first path to the classical limit concerns systems with a very large number of particles, i.e., when $N \rightarrow \infty$. Although in TGP-ST the effective Planck constant $\hbar_{eff}$ (scaled by $N$) increases with $N$, \textbf{quantum fluctuations become negligible compared to macroscopic observations}. This can be understood by analyzing relative uncertainties:
\begin{itemize}
    \item \textbf{Suppression of Relative Uncertainty:} For a system composed of $N$ particles, the total position uncertainty scales as $\Delta X_{total} \propto \sqrt{N}\hbar_{eff}$. However, what matters for classical behavior is the relative uncertainty (e.g., with respect to the system’s size or momentum), which tends to zero: $\Delta X_{rel} \propto 1/N \rightarrow 0$ as $N \rightarrow \infty$. This means that in macroscopic systems, consisting of a huge number of particles, quantum fluctuations become undetectable at the macroscopic level, resulting in observable classical behavior.
    \item \textbf{Dominance of the Geometric-Quantum Term:} In the TGP-Schrödinger equation, for systems with a very large number of particles, the term coupling the wave function to the local geometry ($\lambda(\nabla^2\Phi/\Phi)$) may enforce trajectories close to classical ones. As $N$ increases, the influence of local geometry becomes dominant, steering the system’s dynamics toward a classical description.
\end{itemize}
In this limit, the TGP-Schrödinger equation can be reduced to the Hamilton–Jacobi equation, describing classical trajectories.

\subsection{Strong-Field Limit ($\Phi \rightarrow \infty$)}
The second pathway to the classical limit is especially relevant in regions with extreme spacetime conditions, characterized by very high “density” of generated space, i.e., when $\Phi \rightarrow \infty$. According to our definition (see Section~\ref{sec:StaleDynamiczne}), the effective Planck constant $\hbar(\Phi)$ is inversely proportional to $\sqrt{\Phi}$:
\begin{equation}
    \hbar(\Phi) = \hbar_0 \left(\frac{\Phi_0}{\Phi}\right)^{1/2}
\end{equation}
As a result, when $\Phi$ tends to infinity, $\hbar(\Phi)$ tends to zero ($\hbar(\Phi) \rightarrow 0$).
\begin{itemize}
    \item \textbf{Suppression of Quantum Effects:} A decreasing value of $\hbar(\Phi)$ effectively suppresses all quantum effects, even for single particles, leading to dominant classical behavior. In such regions, phenomena like tunneling, superposition, or uncertainty become practically undetectable.
    \item \textbf{Consistency with General Relativity:} In this limit, where quantum effects vanish, the TGP-ST field equations (especially the gravitational field equations) can reduce to the Einstein equations of General Relativity. This ensures consistency with the success of GR in describing strong gravitational fields.
\end{itemize}
This limit is crucial for understanding the dynamics of objects such as black holes, where TGP-ST predicts the absence of singularities via classical behavior of matter in extreme $\Phi$ regimes.

\subsection{Mathematical Consistency with the Hamilton–Jacobi Equation}
Regardless of the path chosen, the transition to the classical limit can be formally described using standard techniques of quantum mechanics. Considering the semiclassical form of the wave function $\Psi = \rho e^{iS/\hbar_{eff}}$, the TGP-Schrödinger equation in the classical limit reduces to the Hamilton–Jacobi equation:
\begin{equation}
    \frac{\partial S}{\partial t} + \frac{(\nabla S)^2}{2m} + V_{eff} = 0
\end{equation}
The key mechanism of reduction is the vanishing of the term $\frac{\hbar_{eff}^2}{2m}\frac{\nabla^2\rho}{\rho}$ (known as quantum pressure), which represents quantum effects. In the statistical limit, even if $\hbar_{eff}$ is large, the term $\frac{\nabla^2\rho}{\rho}$ (reflecting fluctuations in probability density) decreases faster than $\hbar_{eff}^2$ increases, due to suppression of fluctuations by a large $N$. In the strong-field limit, the disappearance of this term is a direct consequence of $\hbar_{eff} \rightarrow 0$.

\begin{table}[h!]
    \centering
    \small 
    \setlength{\tabcolsep}{3.5pt}
    \begin{tabular}{|>{\centering\arraybackslash}p{2.5cm}|p{4.2cm}|p{4.2cm}|p{3.8cm}|}
        \hline
        \textbf{Mechanism} & \textbf{Dominant Condition} & \textbf{Effect} & \textbf{Example} \\
        \hline
        Statistical ($N \rightarrow \infty$) & $\Delta X_{rel} \propto 1/N \rightarrow 0$ & Suppression of relative fluctuations & Macroscopic systems (planets, billiard balls) \\
        \hline
        Strong Field ($\Phi \rightarrow \infty$) & $\hbar(\Phi) \propto (\Phi_0/\Phi)^{1/2} \rightarrow 0$ & Suppression of quantum effects & Black hole horizons, early Universe \\
        \hline
    \end{tabular}
    \caption{Comparison of two pathways to the classical limit in TGP-ST.}
    \label{tab:ClassicalLimits}
\end{table}

---

\section{Key Physical Consequences and Resolved Problems}
\label{sec:Konsekwencje}

The Generated Space Theory (TGP-ST) offers new perspectives on many long-standing problems in theoretical physics and cosmology, stemming directly from its foundational assumptions regarding the emergent nature of space and the dynamical character of fundamental constants.

\subsection{Elimination of Singularities}
One of the most pressing problems in General Relativity (GR) is the existence of singularities, where matter density and spacetime curvature become infinite (e.g., at the centers of black holes or at the Big Bang). In TGP-ST, the concept of singularities is fundamentally forbidden. Since space is generated by mass, its \textbf{compression to a point of zero volume is inherently impossible}.

Within TGP-ST, in the vicinity of extremely dense objects (where $\Phi \rightarrow \infty$), the speed of light $c(\Phi)$ approaches zero ($c(\Phi) \propto 1/\sqrt{\Phi}$). This means that light (and all information) becomes effectively “frozen” and cannot escape, elegantly explaining the mechanism of the event horizon. Furthermore, in such regions, the effective Planck constant $\hbar(\Phi)$ also tends toward zero, leading to \textbf{classical behavior of matter} (see Section~\ref{sec:GranicaKlasyczna}). This prevents quantum collapse and infinite densities, suggesting the existence of a finite, though extremely dense region where quantum effects are suppressed and light speed vanishes, effectively eliminating singularities.

\subsection{Nature of Dark Energy and Expansion of the Universe}
The problem of dark energy, which is responsible for the accelerated expansion of the Universe, remains one of the greatest puzzles in cosmology. In TGP-ST, the scalar field $\Phi$ and its self-interaction potential $V(\Phi)$ play a crucial role in this matter.

In the vacuum (i.e., the ground state where $\Phi \rightarrow \Phi_0$), the potential $V(\Phi_0)$ acts as an \textbf{effective cosmological constant ($\Lambda_{eff}$)}. Its value naturally arises from the dynamics of the space-generating field. Moreover, TGP-ST offers a new perspective on the expansion of the Universe itself. Rather than invoking only abstract vacuum energy, accelerated expansion can be understood as a \textbf{macroscopic effect of local generation of additional space by mass concentrations} (e.g., in galaxy clusters). These locally generated spatial volumes add up, leading to the global expansion of the Universe, in accordance with observations.

\subsection{Cosmological Tensions (Hubble Tension)}
There are currently significant discrepancies in the measurements of the Universe’s expansion rate (the Hubble constant $H_0$), depending on the methods and scales used. Measurements based on the early Universe (e.g., the Cosmic Microwave Background) differ from those based on the late Universe (e.g., supernovae). TGP-ST offers a potential explanation for this discrepancy.

Standard measurement methods assume a constant speed of light $c_0$. However, in TGP-ST, $c(\Phi)$ is variable. If the local density of generated space ($\Phi$) or its evolution over cosmic time affects the speed of light, then \textbf{our current measurements of distance and expansion rate are misinterpreted}. For instance, a lower $c(\Phi)$ in regions of higher matter density (or in the early Universe) could make objects appear more distant than predicted by standard models, potentially alleviating the Hubble tension.

\subsection{A New Perspective on the Nature of Gravity}
TGP-ST redefines gravity from the ground up. It is not merely geometry curved by matter, but rather a dynamic process in which \textbf{material objects actively create the space in which they exist and interact}. The variability of the gravitational constant $G(\Phi)$ (see Section~\ref{sec:StaleDynamiczne}) directly reflects this, where the “strength” of gravity is modulated by the “density” of the generated space.

\subsection{Quantum Uncertainty and Fluctuations}
In TGP-ST, quantum uncertainty and virtual particles gain a fundamental physical justification within the dynamics of space itself. \textbf{Quantum fluctuations of the generated space} are the direct source of positional uncertainty and manifest as virtual particles. This explains their ubiquity and influence on physics even in seemingly empty regions, as they are an inherent feature of the dynamic spatial network.

---

\section{Testable Predictions}
\label{sec:Testy}

The Theory of Generated Space (TGP-ST), with its unique assumptions regarding the dynamic nature of spacetime and the variability of fundamental constants, leads to a number of testable predictions that differ from those of the Standard Model and General Relativity. Verifying these predictions is key to validating the TGP-ST framework.

\subsection{Anomalies in Quantum Systems and Planck Constant Fluctuations}
According to TGP-ST, the effective Planck constant $\hbar(\Phi)$ is variable and depends on the local "density" of the generated space. This leads to several measurable consequences:
\begin{itemize}
    \item \textbf{Modifications of the uncertainty principle:} The Heisenberg uncertainty principle, $\Delta x \Delta p \ge \hbar(\Phi)/2$, would depend on the local value of the $\Phi$ field. In regions with higher $\Phi$ (e.g., near massive bodies), $\hbar(\Phi)$ would be smaller, suggesting that quantum uncertainties may be locally reduced. Precision experiments measuring quantum fluctuations in environments with gravitational gradients (and thus $\Phi$ gradients) could detect such anomalies.
    \item \textbf{Tunneling probability dependence on geometry:} Quantum tunneling probabilities would depend on the local value and fluctuations of the $\Phi$ field. In high-$\Phi$ regions (where $\hbar(\Phi)$ is lower), tunneling may be suppressed, providing a measurable deviation. Such tests could be performed using electron tunneling setups or Bose–Einstein condensates.
    \item \textbf{Spectroscopic modifications:} Atomic energy levels and their emission/absorption spectra depend on $\hbar$. If $\hbar$ is variable, then energy levels $E_n$ will depend on the local $\Phi$ value, potentially leading to subtle shifts in atomic spectra. These shifts could be measurable across different gravitational environments (e.g., Earth vs. space, or near massive bodies, assuming sufficient precision).
\end{itemize}

\subsection{Astrophysical and Cosmological Observations}
The large-scale consequences of TGP-ST generate predictions that can be tested via astronomical observations:
\begin{itemize}
    \item \textbf{Modifications in the Universe’s expansion rate:} The variable speed of light $c(\Phi)$ and gravitational constant $G(\Phi)$ directly influence the Friedmann equations and cosmological dynamics. TGP-ST may help resolve (or at least clarify) the \textbf{Hubble tension}, predicting that $H_0$ measurements from different epochs or scales could differ due to varying average values of $\Phi$ and $c(\Phi)$ across those epochs/regions. Accurate modeling of $H(z)$ evolution in TGP-ST will be essential.
    \item \textbf{Properties of black holes and compact objects:} In TGP-ST, black holes do not feature singularities, and $c(\Phi)$ approaches zero at the horizon. This may lead to subtle deviations in gravitational wave signals from black hole mergers (e.g., in the ringdown phase), potentially containing signatures of the $\Phi$ field. Additionally, extreme environments near neutron stars and active galactic nuclei may show deviations from GR predictions.
    \item \textbf{Light propagation in strong gravitational fields:} Since $c(\Phi)$ varies, light traveling through very dense regions of the universe (e.g., galaxy cluster cores) may propagate at a speed different from that in vacuum. This could be measurable via precision astrometry or gravitational lensing effects.
    \item \textbf{Additional gravitational wave polarizations:} As a scalar–tensor theory, TGP-ST may predict the presence of additional gravitational wave polarizations (e.g., scalar modes), absent in GR. Future gravitational wave detectors may be able to detect such signatures.
\end{itemize}

\subsection{Potential Laboratory Experiments}
Although TGP-ST concerns fundamental forces, some of its consequences may be testable under controlled laboratory conditions—especially those related to variability of $\hbar(\Phi)$ and $c(\Phi)$:
\begin{itemize}
    \item \textbf{Measuring local variability of $c$:} In variable speed of light theories, it is in principle possible that $c$ may vary slightly in the presence of very massive objects (e.g., deep underground—though the effect would be extremely small), or in lab-generated regions of variable $\Phi$ density.
    \item \textbf{Tests of the equivalence principle:} Due to coupling between $\Phi$ and matter, subtle violations of the equivalence principle could arise. These might be detected by experiments such as Eötvös or MICROSCOPE, which look for differences in free-fall acceleration among different substances.
\end{itemize}

Verification of these predictions constitutes a long-term research objective, requiring both the development of high-precision measurement techniques and further theoretical refinement of TGP-ST.

---

\section{Discussion and Open Problems}
\label{sec:OtwarteProblemy}

This paper presents an outline of the Theory of Generated Space (TGP-ST) as a potential and consistent framework in the quest to unify gravity and quantum mechanics. The author is fully aware that the theory, in its current form, \textbf{does not answer all questions}; indeed, its formulation often \textbf{raises more new questions than it provides definitive answers}. The aim of this preprint is not to present a final, fully proven and complete theory, but rather \textbf{to propose the fundamental idea that viewing space as a mass-generated entity has deep physical meaning and deserves further scientific investigation and development}. This work is intended as an invitation for discussion and collaboration to explore the implications of TGP-ST.

As with any new fundamental theory, TGP-ST involves several challenges that require further in-depth research and development. The key open problems are outlined below.

\subsection{Lorentz Covariance and the Concept of the Speed of Light}
In TGP-ST, the speed of light $c(\Phi)$ is a dynamic field dependent on the local “density” of generated space. This variability implies a modification of the fundamental principle of \textbf{Lorentz invariance}, which is a cornerstone of Special Relativity (SR) and, indirectly, General Relativity (GR). 

In standard SR, $c$ is an absolute constant for all inertial frames of reference. In TGP-ST, $c(\Phi)$ is still the local maximal speed of information propagation, but its \textit{numerical value} may vary in space and time. This requires a generalization of the Lorentz transformations, which would depend on the $\Phi$ field. While this might seem problematic at first, it is a direct consequence of the emergent and dynamic concept of space. The challenge is to develop a full covariant formalism describing field dynamics in a spacetime with a locally variable speed of light, while preserving causality.

\subsection{Energy and Momentum Conservation}
The variability of the speed of light $c(\Phi)$ and gravitational constant $G(\Phi)$ in TGP-ST has direct implications for the \textbf{conservation laws of energy and momentum}. In standard physics, these principles are tightly connected to spacetime symmetries (via Noether's theorem). If the metric and fundamental constants are dynamic, traditional definitions of conservation may need to be generalized.

As previously noted, the energy of a system may not be globally conserved if the $\Phi$ field evolves, potentially leading to an equation of the form $dE/dt = -E/(2\Phi) \cdot d\Phi/dt$. This calls for a detailed analysis of whether the theory can define generalized conserved quantities (e.g., at the level of a combined matter-$\Phi$ pseudo-stress tensor), or whether it accepts that energy may be freely exchanged with the “fabric” of space, a possibility with far-reaching implications.

\subsection{UV Renormalization and Quantum Gravity}
As with many gravity theories, the classical formulation of TGP-ST (and any attempt at perturbative quantization) is likely to be \textbf{non-renormalizable in the ultraviolet (UV) regime}. That is, straightforward perturbative methods may lead to divergences that cannot be canceled with a finite number of counterterms.

TGP-ST will require the use of non-perturbative approaches to quantum gravity. Potential directions include:
\begin{itemize}
    \item \textbf{Asymptotic safety:} The hypothesis that the theory becomes well-behaved at high energies, reaching a fixed point in the space of coupling constants.
    \item \textbf{Emergent quantum gravity:} A framework where both quantum gravity and spacetime emerge from even more fundamental degrees of freedom, with TGP-ST serving as an effective theory.
\end{itemize}
This is a challenge common to most approaches to quantum gravity and is not unique to TGP-ST.

\subsection{Exact Solutions to the Field Equations}
At present, complete analytical solutions to the coupled Einstein–TGP-ST equations and the $\Phi$ field equation are known only for very simple geometries (e.g., Schwarzschild metric in vacuum, as discussed). Realistic scenarios (e.g., cosmological evolution, astrophysical processes) will require \textbf{complex numerical solutions}. Accurate observational predictions—especially in the context of the Hubble tension—will require precise modeling of light propagation in the dynamically generated TGP-ST spacetime.

\subsection{Future Research Directions}
Further research on TGP-ST should focus on:
\begin{itemize}
    \item Developing a complete quantum formalism of the theory, including quantization of the $\Phi$ field and its couplings.
    \item Analyzing the theory’s stability and consistency across different energy regimes.
    \item Formulating more detailed astrophysical and laboratory predictions that can be compared to observational data.
    \item Investigating whether TGP-ST can naturally incorporate other Standard Model forces within its emergent framework.
\end{itemize}
Despite these challenges, TGP-ST offers a unique and powerful framework for rethinking the fundamental aspects of physics, pointing toward a promising path for unification.


---

\section{Conclusion}
\label{sec:Podsumowanie}

In this work, we presented an outline of the \textbf{Theory of Generated Space (TGP-ST)}, a radical approach to fundamental forces in physics that proposes a unification of gravity and quantum mechanics. The central premise of TGP-ST is the idea that \textbf{space is not a passive backdrop, but a dynamic emergent entity generated directly by the presence of mass and energy}. This revolutionary concept leads to the rejection of the traditional notion of vacuum, replacing it with a dynamic network of generated space, whose "density" is described by the scalar field $\Phi$. In this context, virtual particles gain a new interpretation as local manifestations of quantum fluctuations of the generated space itself.

A key feature of TGP-ST is the fundamental \textbf{variability of physical “constants”}: the speed of light $c(\Phi)$, Planck’s constant $\hbar(\Phi)$, and the gravitational constant $G(\Phi)$ become fields dependent on the local value of $\Phi$. We have shown that these dynamic dependencies can preserve the \textbf{constancy of the Planck length $\ell_P$}, providing an elegant and consistent result within the principles of quantum gravity.

This theory offers potential and consistent resolutions to many unresolved problems in modern physics:
\begin{itemize}
    \item \textbf{Elimination of singularities} in black holes and at the beginning of the Universe, due to the fact that $c(\Phi)$ and $\hbar(\Phi)$ tend to zero in regions of extremely high $\Phi$.
    \item A new interpretation of \textbf{dark energy} as an inherent property of the potential $V(\Phi)$ and a perspective on the \textbf{expansion of the Universe} as an effect of local space generation by matter concentrations.
    \item A potential resolution of the \textbf{Hubble tension} through the variability of $c(\Phi)$ on cosmological scales.
    \item A novel, physically grounded interpretation of \textbf{quantum uncertainty} and \textbf{virtual particles}, rooted in fluctuations of the generated space itself.
\end{itemize}

We presented modified equations of motion for gravity and the $\Phi$ field, as well as the Schrödinger-TGP equation with an emergent Planck constant. We discussed two distinct paths to the classical limit, ensuring a smooth transition from the microscopic to the macroscopic world. Most importantly, TGP-ST produces \textbf{concrete and testable predictions} in the domains of quantum physics (e.g., modifications to the uncertainty principle) and astrophysics (e.g., signatures in gravitational waves, deviations in cosmological distance measurements).

Although TGP-ST presents several challenges—such as Lorentz covariance, energy conservation, and full quantum renormalization—its internal consistency and capacity to address fundamental issues make it a \textbf{promising direction for further investigation}. The Theory of Generated Space represents a bold step toward a unified description of the Universe, where space and matter are inseparably intertwined in a dynamic dance of creation.

---


\begin{thebibliography}{99} 

\bibitem{Einstein1915} A. Einstein, "Die Feldgleichungen der Gravitation," \textit{Sitzungsberichte der Preussischen Akademie der Wissenschaften zu Berlin}, 844-847 (1915).
\bibitem{Schrodinger1926} E. Schr\"odinger, "An Undulatory Theory of the Mechanics of Atoms and Molecules," \textit{Physical Review} \textbf{28}(6), 1049-1070 (1926).
\bibitem{Heisenberg1927} W. Heisenberg, "Ueber den anschaulichen Inhalt der quantentheoretischen Kinematik und Mechanik," \textit{Zeitschrift f\"ur Physik} \textbf{43}(3-4), 172-198 (1927).
\bibitem{Dirac1928} P. A. M. Dirac, "The Quantum Theory of the Electron," \textit{Proceedings of the Royal Society of London. Series A, Containing Papers of a Mathematical and Physical Character} \textbf{117}(778), 610-624 (1928).

\bibitem{MTW} C. W. Misner, K. S. Thorne, J. A. Wheeler, \textit{Gravitation}, W. H. Freeman (1973).
\bibitem{Wald1984} R. M. Wald, \textit{General Relativity}, University of Chicago Press (1984).

\bibitem{BransDicke1961} C. Brans and R. H. Dicke, "Mach's Principle and a Relativistic Theory of Gravitation," \textit{Physical Review} \textbf{124}(3), 925-935 (1961).
\bibitem{Moffat2004} J. W. Moffat, "A New Theory of Gravity," \textit{Phys. Rev. D} \textbf{70}, 124009 (2004) [\url{arXiv:gr-qc/0010078}].
\bibitem{Magueijo2003} J. Magueijo, "Covariant theories of varying c," \textit{Rept. Prog. Phys.} \textbf{66}, 2025-2068 (2003) [\url{arXiv:astro-ph/0305457}]. 

\bibitem{Perlmutter1999} S. Perlmutter et al. (Supernova Cosmology Project), "Measurements of Omega and Lambda from 42 High-Redshift Supernovae," \textit{Astrophysical Journal} \textbf{517}(2), 565-586 (1999) [\url{arXiv:astro-ph/9812133}]. % Użycie \url{}
\bibitem{Riess1998} A. G. Riess et al. (Supernova Search Team), "Observational Evidence from Supernovae for an Accelerating Universe and a Cosmological Constant," \textit{Astronomical Journal} \textbf{116}(3), 1009-1038 (1998) [\url{arXiv:astro-ph/9805201}]. 
\bibitem{Planck2020} Planck Collaboration, "Planck 2018 results. VI. Cosmological parameters," \textit{Astronomy \& Astrophysics} \textbf{641}, A6 (2020) [\url{arXiv:1807.06209}]. 

\bibitem{Weinberg1979} S. Weinberg, "Ultraviolet divergences in quantum theories of gravitation," in \textit{General Relativity: An Einstein Centenary Survey}, ed. S.W. Hawking and W. Israel, Cambridge University Press, 790-831 (1979).

\bibitem{Verde2019} L. Verde, T. Treu, A. G. Riess, "Tensions between the Early and the Late Universe," \textit{Nature Astronomy} \textbf{3}, 891-895 (2019) [\url{arXiv:1907.10625}].

\end{thebibliography}

\end{document}