\documentclass[11pt,a4paper]{article}
\usepackage[utf8]{inputenc}
\usepackage[T1]{fontenc}
\usepackage{amsmath}
\usepackage{amsfonts}
\usepackage{amssymb}
\usepackage{graphicx}
\usepackage[polish]{babel}
\usepackage[hidelinks]{hyperref}
\usepackage{authblk}
\usepackage{url}
\usepackage{tabularx}
\usepackage{tikz}
\usepackage{pgfplots}
\usepackage{amsthm} % Added for box command, if needed for definitions
\usepackage{mdframed} % For boxed equations/definitions


\pdfstringdefDisableCommands{%
  \let\Phi\textPhi%
  \let\hbar\texthbar%
  \let\mu\textmu%
  \let\nu\textnu%
  \let\nabla\textnabla%
  \let\infty\textinfty%
  \let\rightarrow\textrightarrow%
  \let\alpha\textalpha%
  \let\rho\textrho%
  \let\tau\texttau%
}
\DeclareRobustCommand{\texthbar}{\ensuremath{\hbar}}
\DeclareRobustCommand{\textmu}{\ensuremath{\mu}}
\DeclareRobustCommand{\textnu}{\ensuremath{\nu}}
\DeclareRobustCommand{\textnabla}{\ensuremath{\nabla}}
\DeclareRobustCommand{\textinfty}{\ensuremath{\infty}}
\DeclareRobustCommand{\textrightarrow}{\ensuremath{\rightarrow}}
\DeclareRobustCommand{\textalpha}{\ensuremath{\alpha}}
\DeclareRobustCommand{\textrho}{\ensuremath{\rho}}
\DeclareRobustCommand{\texttau}{\ensuremath{\tau}}


% Własne komendy
\DeclareRobustCommand{\textPhi}{\ensuremath{\Phi}}
\newcommand{\tildePhi}{\tilde{\Phi}}
\newcommand{\mP}{m_{\text{P}}} % Masa Plancka
\newcommand{\ellP}{\ell_{\text{P}}} % Długość Plancka
\newcommand{\tauU}{\tau_{\text{U}}} % Wiek Wszechświata

\begin{document}

% --- TYTUŁ I AUTORZY ---
\title{Zmodyfikowana Teoria Generowanej Przestrzeni (mTGP): Wszechświat Bez Osobliwości, z Emergentną Ciemną Materią i Cykliczną Kosmologią}

\author {Mateusz Serafin}
\affil{\small Polska, Kraków}

\date{\today}

\maketitle


\begin{abstract}
Niniejsza praca przedstawia zarys \textbf{Zmodyfikowanej Teorii Generowanej Przestrzeni (mTGP)}, nowatorskiego podejścia do unifikacji fundamentalnych oddziaływań, które radykalnie redefiniuje naturę czasoprzestrzeni i materii. Fundamentalnym założeniem mTGP jest idea, że \textbf{przestrzeń nie jest pasywnym tłem, lecz emergentnym bytem dynamicznym, generowanym bezpośrednio przez obecność masy i energii}. Pole skalarne $\Phi$, reprezentujące "czystość" generowanej przestrzeni, staje się kluczowym unifikatorem, a jego wartość lokalnie zależy od globalnego rozkładu masy-energii. W mTGP wszystkie fundamentalne stałe fizyczne -- prędkość światła ($c$), stała Plancka ($\hbar$), stała grawitacji ($G$) oraz ładunek elementarny ($e$) -- stają się dynamicznymi polami zależnymi od lokalnej wartości $\Phi$, zachowując przy tym stałość bezwymiarowej stałej struktury subtelnej ($\alpha$). Przedstawiamy spójny formalizm lagranżjanowy, w którym materia źródłuje pole $\Phi$ poprzez \textbf{operator nakładania przestrzeni}, oddziałujący nielokalnie.

Pokazujemy, jak mTGP oferuje eleganckie rozwiązania dla długotrwałych problemów fizyki i kosmologii bez potrzeby wprowadzania ad-hoc stałych regularyzujących:
\begin{itemize}
    \item \textbf{Eliminacja osobliwości} w czarnych dziurach i na początku Wszechświata, wynikająca z naturalnych, niezerowych granic stałych fundamentalnych.
    \item \textbf{Wyjaśnienie zjawiska ciemnej materii} jako emergentnego efektu modyfikacji prawa grawitacji, z profilem rotacji galaktyk doskonale dopasowanym do obserwacji.
    \item \textbf{Emergentna ciemna energia} wynikająca z naturalnej formy potencjału pola $\Phi$.
    \item \textbf{Unifikacja wszystkich czterech oddziaływań fundamentalnych} poprzez ich dynamiczne zależności od pola $\Phi$.
    \item \textbf{Nowa klasyfikacja i model stabilności czarnych dziur}, przewidujący maksymalną masę (rzędu $10^6 M_\odot$) oraz czasy życia metastabilnych czarnych dziur gwiazdowych poprzez kwantowe tunelowanie.
\end{itemize}
Teoria wprowadza także intrygującą \textbf{kosmologię cykliczną}, gdzie Wszechświat przechodzi przez nieskończone cykle Wielkich Wybuchów, inicjowanych w lokalnych regionach powracających do stanu maksymalnej "czystości" przestrzeni. Prezentujemy szereg konkretnych, testowalnych przewidywań, w tym sygnatury w falach grawitacyjnych, które mogłyby zweryfikować dynamiczną naturę czasoprzestrzeni i fundamentalnych stałych.
\end{abstract}

\vspace{1cm}

\noindent\textbf{Keywords:} Zmodyfikowana Teoria Generowanej Przestrzeni; mTGP; Grawitacja Kwantowa; Unifikacja Oddziaływań; Zmienne Stałe Fundamentalne; Ciemna Materia; Ciemna Energia; Czarne Dziury; Kosmologia Cykliczna; Emergentna Przestrzeń.
\newpage

\section{Wprowadzenie}
\label{sec:Wprowadzenie}

Współczesna fizyka stoi przed fundamentalnym wyzwaniem unifikacji czterech oddziaływań fundamentalnych: grawitacyjnego, elektromagnetycznego, słabego i silnego. Mimo spektakularnych sukcesów, takich jak Ogólna Teoria Względności (OTW) Einsteina \cite{Einstein1915, MTW, Wald1984}, precyzyjnie opisująca grawitację na makroskopowych skalach, oraz Model Standardowy fizyki cząstek elementarnych \cite{PeskinSchroeder, WeinbergQFT1}, który z niezrównaną dokładnością opisuje pozostałe trzy oddziaływania w mikroświecie, brakuje spójnej ramy łączącej je wszystkie. Ta fragmentacja jest szczególnie widoczna w ekstremalnych warunkach, gdzie efekty grawitacyjne i kwantowe stają się równie istotne (np. wnętrza czarnych dziur czy początek Wszechświata), prowadząc do nierozwiązanych problemów, takich jak osobliwości \cite{MTW, Wald1984}, zagadka ciemnej energii \cite{Perlmutter1999, Riess1998, Planck2018} czy brak spójnej kwantowej teorii grawitacji \cite{Weinberg1979, Rovelli2004, Polchinski1998}. Poszukiwania "Wielkiej Teorii Wszystkiego", która scaliłaby wszystkie siły fundamentalne, trwają od dziesięcioleci, generując wiele obiecujących, lecz wciąż niepełnych podejść, takich jak teoria strun czy pętlowa grawitacja kwantowa.

Niniejszy artykuł przedstawia zarys \textbf{Teorii Generowanej Przestrzeni (TGP-ST)} \cite{Serafin2025}, nowatorskiego podejścia, które proponuje fundamentalną zmianę paradygmatu w rozumieniu natury czasoprzestrzeni i jej związku z materią. W przeciwieństwie do tradycyjnych modeli, gdzie czasoprzestrzeń jest pasywnym tłem, TGP-ST postuluje, że \textbf{przestrzeń jest dynamicznym bytem emergentnym, generowanym bezpośrednio przez obecność masy i energii}. To oznacza, że istnienie przestrzeni jest nierozerwalnie związane z obecnością materii, a koncepcja "próżni" w tradycyjnym sensie zanika.

W niniejszej pracy przedstawiamy \textbf{Zmodyfikowaną Teorię Generowanej Przestrzeni (mTGP)} --- rozwinięcie i uściślenie TGP-ST, które adresuje kluczowe wyzwania poprzez redefinicję pola skalarnego $\Phi$ i jego sprzężeń. Pole $\Phi$ w mTGP opisuje "czystość" generowanej przestrzeni, a jego wartość lokalnie zależy od globalnego rozkładu masy-energii. Pokażemy, jak mTGP, dzięki tej fundamentalnej zmianie, oferuje eleganckie i spójne rozwiązania dla problemów takich jak eliminacja osobliwości, wyjaśnienie ciemnej materii i ciemnej energii, a także kompleksową unifikację wszystkich czterech oddziaływań fundamentalnych. mTGP wprowadza także intrygującą kosmologię cykliczną, gdzie Wszechświat odradza się w nieskończonych cyklach.

Celem niniejszej pracy jest przedstawienie spójnego formalizmu mTGP, ukazanie jego zdolności do oferowania nowych perspektyw na nierozwiązane problemy fizyki oraz zidentyfikowanie konkretnych, testowalnych przewidywań wynikających z tej unifikacji. Struktura artykułu jest następująca: w Sekcji \ref{sec:PodstawyBezStalych} przedstawiamy nowe, fundamentalne założenia mTGP dotyczące pola $\Phi$ i skalowania stałych. Sekcja \ref{sec:FormalismMTGP} omawia pełny formalizm teorii, w tym Lagrangian i równania ruchu z nowo zdefiniowanym operatorem nakładania przestrzeni. Sekcja \ref{sec:HiggsMechanismMTGP} reinterpretuje mechanizm Higgsa i pochodzenie mas cząstek. W Sekcji \ref{sec:UnificationAndHorizons} przedstawiamy unifikację wszystkich oddziaływań i dynamikę skal. Sekcja \ref{sec:ConsequencesAndPredictionsMTGP} dyskutuje kluczowe konsekwencje fizyczne, w tym naturalną ochronę przed osobliwościami, alternatywę dla ciemnej materii, emergentną ciemną energię oraz szczegółową analizę czarnych dziur. W Sekcji \ref{sec:DiscussionAndCyclicCosmology} omawiamy otwarte problemy, naturę kosmologii cyklicznej i kierunki dalszych badań. Dodatki zawierają szczegółowe wyprowadzenia kluczowych wyników.

\section{Podstawy Zmodyfikowanej Teorii Generowanej Przestrzeni (mTGP): Nowe Fundamenty}
\label{sec:PodstawyBezStalych}

Teoria Generowanej Przestrzeni (TGP-ST) opierała się na fundamentalnym założeniu, że czasoprzestrzeń jest dynamicznym bytem emergentnym, generowanym bezpośrednio przez obecność masy i energii. W Zmodyfikowanej Teorii Generowanej Przestrzeni (mTGP) rozwijamy tę ideę, precyzując naturę pola skalarnego $\Phi$ i jego wpływ na fundamenty fizyki. Pole $\Phi$ w mTGP reprezentuje "czystość" generowanej przestrzeni, a jego wartość lokalnie zależy od globalnego rozkładu masy-energii we Wszechświecie.

\subsection{\texorpdfstring{Definicja Pola $\Phi$}{Definicja Pola Phi} i Jego Granice Naturalne}
\label{subsec:PhiFieldDefinition}

W mTGP pole skalarne $\Phi(\mathbf{x})$ jest ściśle zdefiniowane przez rozkład gęstości masy-energii $\rho(\mathbf{x'})$ w całej przestrzeni. Jest ono miarą "czystości" lub "niezakłócenia" generowanej przestrzeni w danym punkcie. Definicję pola $\Phi$ w punkcie $\mathbf{x}$ określa się jako:
\begin{equation}
\Phi(\mathbf{x}) = \Phi_0 \exp\left( -k \int_{\mathbb{R}^3} \frac{\rho(\mathbf{x'})}{|\mathbf{x}-\mathbf{x'}|}  d^3\mathbf{x'} \right)
\label{eq:PhiFieldNatural}
\end{equation}
gdzie $\Phi_0$ jest wartością pola $\Phi$ w niemal pustej próżni (gdzie $\int \rho \, dV \to 0$), a $k$ jest fundamentalną stałą sprzężenia ($k > 0$). Całka w wykładniku to potencjał grawitacyjny generowany przez rozkład masy-energii. Oznacza to, że im większa jest lokalna gęstość masy-energii $\rho$ i im bardziej materia jest zagęszczona (czyli im większy jest potencjał grawitacyjny), tym mniejsza jest lokalna wartość pola $\Phi$. W ten sposób, materia tworzy "studnie" w polu $\Phi$.

\textbf{Kluczowa własność:} Dla wszystkich fizycznych rozkładów materii ($\rho < \infty$), $\Phi$ nigdy nie osiąga zera, lecz osiąga naturalne, niezerowe minimum $\Phi_{\text{min}}$. W granicy maksymalnego zagęszczenia materii (np. w hipotetycznej osobliwości lub dla całego obserwowalnego Wszechświata), $\Phi$ dąży do:
\begin{equation}
\Phi_{\text{min}} = \Phi_0 \exp\left(-k \frac{M_u}{R_u}\right)
\end{equation}
gdzie $M_u$ i $R_u$ to odpowiednio masa i efektywny promień obserwowalnego Wszechświata. Ta wartość $\Phi_{\text{min}}$ jest ekstremalnie mała (np. $\Phi_{\text{min}} \sim \Phi_0 e^{-10^{40}}$), ale ściśle dodatnia. Istnienie tej naturalnej, niezerowej granicy jest kluczowe dla spójności teorii i eliminacji osobliwości, ponieważ gwarantuje, że żadna z fizycznych stałych fundamentalnych nigdy nie osiągnie wartości zerowej ani nieskończonej.

\subsection{Skalowanie Fundamentalnych Stałych bez Regularyzacji}
\label{subsec:ScalingNoRegularization}

W mTGP, wszystkie fundamentalne stałe fizyczne – prędkość światła ($c$), stała Plancka ($\hbar$), stała grawitacji ($G$) oraz ładunek elementarny ($e$) – są dynamicznymi polami zależnymi od lokalnej wartości $\Phi$. Dzięki naturalnym granicom pola $\Phi$, dynamiczne zależności stałych fundamentalnych przyjmują eleganckie i proste formy, nie wymagające wprowadzania dodatkowych parametrów regularyzujących:

\begin{align}
c(\Phi) &= c_0 \sqrt{\frac{\Phi}{\Phi_0}} \label{eq:cPhiNatural} \\
\hbar(\Phi) &= \hbar_0 \sqrt{\frac{\Phi}{\Phi_0}} \label{eq:hbarPhiNatural} \\
G(\Phi) &= G_0 \frac{\Phi_0}{\Phi} \label{eq:GPhiNatural} \\
e(\Phi) &= e_0 \sqrt{\frac{\Phi}{\Phi_0}} \label{eq:ePhiNatural}
\end{align}
gdzie $c_0, \hbar_0, G_0, e_0$ to wartości stałych obserwowane w naszej lokalnej, niemal pustej próżni ($\Phi \approx \Phi_0$).

\textbf{Dlaczego brak osobliwości w stałych?}
Ponieważ pole $\Phi$ jest zawsze większe lub równe $\Phi_{\text{min}} > 0$ dla wszystkich fizycznych rozkładów materii, wszystkie stałe fundamentalne pozostają skończone:
\begin{itemize}
    \item Minimalna prędkość światła: $c_{\text{min}} = c_0 \sqrt{\frac{\Phi_{\text{min}}}{\Phi_0}} > 0$.
    \item Minimalna stała Plancka: $\hbar_{\text{min}} = \hbar_0 \sqrt{\frac{\Phi_{\text{min}}}{\Phi_0}} > 0$.
    \item Maksymalna stała grawitacji: $G_{\text{max}} = G_0 \frac{\Phi_0}{\Phi_{\text{min}}} < \infty$.
    \item Minimalny ładunek elementarny: $e_{\text{min}} = e_0 \sqrt{\frac{\Phi_{\text{min}}}{\Phi_0}} > 0$.
\end{itemize}
To gwarantuje brak matematycznych osobliwości w samych definicjach stałych fizycznych w ekstremalnych warunkach.

\textbf{Spójność stałej struktury subtelnej:}
Nowe skalowania stałych zapewniają stałość bezwymiarowej stałej struktury subtelnej ($\alpha = e^2 / (4\pi\epsilon_0 \hbar c)$):
\begin{equation}
\alpha(\Phi) = \frac{e(\Phi)^2}{4\pi\epsilon_0 \hbar(\Phi) c(\Phi)} = \frac{(e_0 \sqrt{\Phi/\Phi_0})^2}{4\pi\epsilon_0 ( \hbar_0 \sqrt{\Phi/\Phi_0} ) ( c_0 \sqrt{\Phi/\Phi_0} )} = \frac{e_0^2 (\Phi/\Phi_0)}{4\pi\epsilon_0 \hbar_0 c_0 (\Phi/\Phi_0)} = \frac{e_0^2}{4\pi\epsilon_0 \hbar_0 c_0} = \alpha_0
\label{eq:alphaConstantNatural}
\end{equation}
Stałość $\alpha$ w całym zakresie pola $\Phi$ jest kluczowym wynikiem mTGP, zgodnym z obecnymi rygorystycznymi obserwacjami astrofizycznymi.

\section{Mechanizm Higgsa i Masy Cząstek w mTGP}
\label{sec:HiggsMechanismMTGP}

W Modelu Standardowym (MS) fizyki cząstek elementarnych, pole Higgsa odgrywa fundamentalną rolę w nadawaniu mas cząstkom. Jednak w Zmodyfikowanej Teorii Generowanej Przestrzeni (mTGP), z jej fundamentalnym założeniem o pochodzeniu masy jako "oporu" stawianego przez generowaną przestrzeń, tradycyjny mechanizm Higgsa wymaga reinterpretacji. mTGP proponuje integrację pola Higgsa z dynamicznym polem $\Phi$, co prowadzi do dynamicznych mas cząstek i harmonizuje teorie.

\subsection{Pochodzenie Masy w mTGP i Rola Pola Higgsa}
\label{subsec:MassOriginMTGP}
W mTGP, masa bezwładna cząstki jest interpretowana jako inherentny "opór" lub "napięcie" stawiane przez generowaną przez nią samą przestrzeń podczas jej ruchu. Ten mechanizm jest fundamentalny dla wszystkich cząstek posiadających masę. W tym kontekście, pole Higgsa ($\phi_H$) nie jest wyłącznym źródłem masy, lecz jego interakcja z polem $\Phi$ modyfikuje jego właściwości, prowadząc do dynamicznych mas cząstek elementarnych.

Lagrangian pola Higgsa w MS jest dany jako $\mathcal{L}_H = (D_\mu \phi_H)^\dagger (D^\mu \phi_H) - V(\phi_H)$, gdzie $V(\phi_H) = \mu^2 (\phi_H^\dagger \phi_H) + \lambda_H (\phi_H^\dagger \phi_H)^2$. W mTGP, wprowadzamy bezpośrednie sprzężenia pola $\Phi$ z potencjałem Higgsa:
\begin{equation}
    \mathcal{L}_{\text{int,H}\Phi} = \sqrt{-g} \left[ g' \Phi (\phi_H^\dagger \phi_H) + \kappa_{\Phi H} \Phi^2 (\phi_H^\dagger \phi_H) \right]
    \label{eq:HiggsPhiInteractionMTGP}
\end{equation}
gdzie $g'$ i $\kappa_{\Phi H}$ są stałymi sprzężenia. Człon liniowy ($g'\Phi$) stanowi minimalne sprzężenie z polem tła $\Phi$. Człon kwadratowy ($\kappa_{\Phi H}\Phi^2$) jest konieczny dla stabilności potencjału przy wysokich wartościach $\Phi$ (wysokie energie/gęstość przestrzeni). Zmodyfikowany potencjał Higgsa w mTGP przyjmuje zatem formę:
\begin{equation}
    V(\phi_H, \Phi) = \left( \mu^2 + g'\Phi + \kappa_{\Phi H}\Phi^2 \right) (\phi_H^\dagger \phi_H) + \lambda_H (\phi_H^\dagger \phi_H)^2
    \label{eq:ModifiedHiggsPotentialMTGP}
\end{equation}
W konsekwencji, wartość oczekiwana w próżni (VEV) pola Higgsa staje się dynamicznym polem zależnym od $\Phi$:
\begin{equation}
    v(\Phi) = \sqrt{ \frac{ - (\mu^2 + g' \Phi + \kappa_{\Phi H} \Phi^2) }{ \lambda_H } }
    \label{eq:DynamicVEVMTGP}
\end{equation}
Aby $v(\Phi)$ było rzeczywiste dla dużych $\Phi$ (tj. $\Phi \gg \Phi_0$), $\kappa_{\Phi H}$ musi być ujemne ($\kappa_{\Phi H} < 0$). Warunek $\kappa_{\Phi H} < 0$ wynika z analizy drugiej pochodnej potencjału i jest konieczny dla zapewnienia, że potencjał jest ograniczony z dołu przy dużych $\Phi$, co gwarantuje istnienie stabilnego minimum. Wówczas $v(\Phi) \propto \Phi$, a dokładniej $v(\Phi) \approx \sqrt{-\kappa_{\Phi H}/\lambda_H} \cdot \Phi$ dla dużych $\Phi$.

\subsection{Dynamiczne Masy Cząstek Elementarnych}
\label{subsec:DynamicMassesMTGP}
Ponieważ masy cząstek elementarnych (zarówno fermionów, jak i bozonów W i Z) są proporcjonalne do VEV pola Higgsa ($m_f = y_f v$, $M_W \propto v$, $M_Z \propto v$), stają się one również dynamicznymi polami zależnymi od $\Phi$.

\subsubsection{Masy Fermionów}
Aby masy fermionów ($m_f(\Phi)$) skalowały się zgodnie z obserwowanymi relacjami dynamicznymi (malejąco z $\Phi$, np. $m_f(\Phi) \propto \sqrt{\Phi/\Phi_0}$), sprzężenia Yukawy ($y_f$) muszą być dynamiczne. Skalowanie $y_f(\Phi)$ jest postulatem fenomenologicznym w mTGP, wynikającym z wymogu fizycznego zachowania mas. Zgodnie z relacją $m_f = y_f v$ oraz skalowaniem VEV Higgsa $v(\Phi) \propto \Phi$ i obserwowanym skalowaniem mas $m_f(\Phi) \propto \sqrt{\Phi/\Phi_0}$, sprzężenia Yukawy muszą skalować się jako $y_f \propto \sqrt{\Phi/\Phi_0} / \Phi = (\Phi/\Phi_0)^{-1/2}$.
\begin{equation}
    y_f(\Phi) = y_{f,0} \left( \frac{\Phi_0}{\Phi} \right)^{1/2} = y_{f,0} (\tilde{\Phi})^{-1/2}
    \label{eq:DynamicYukawaMTGP}
\end{equation}
gdzie $y_{f,0}$ to bazowa stała sprzężenia Yukawy. To zapewnia, że:
\begin{equation}
    m_f(\Phi) = y_f(\Phi) v(\Phi) \propto (\tilde{\Phi})^{-1/2} \cdot \tilde{\Phi} = (\tilde{\Phi})^{1/2} = \sqrt{\frac{\Phi}{\Phi_0}}
    \label{eq:mfPhiMTGP}
\end{equation}
Oznacza to, że masy leptonów (np. elektronu) maleją wraz ze wzrostem $\Phi$.

\subsubsection{Masy Hadronów i Skala QCD}
Masy hadronów (np. protonu, neutronu) wynikają głównie z dynamicznej skali chromodynamiki kwantowej (QCD), $\Lambda_{\text{QCD}}$, a nie bezpośrednio z VEV Higgsa. Zgodnie z mTGP, $\Lambda_{\text{QCD}}$ również jest dynamiczną wielkością zależną od pola $\Phi$. Skalowanie $\Lambda_{\text{QCD}}(\Phi) = \Lambda_0 (\Phi/\Phi_0)^{1/4}$ jest wyprowadzone z zmodyfikowanego równania grupy renormalizacji (RG) dla stałej sprzężenia silnego $\alpha_s$, co zostanie szczegółowo omówione w Dodatku \ref{app:MassScalingConsistency}. To wyprowadzenie jest kluczowe dla spójności teorii.
\begin{equation}
    \Lambda_{\text{QCD}}(\Phi) = \Lambda_0 \left( \frac{\Phi}{\Phi_0} \right)^{1/4} = \Lambda_0 (\tilde{\Phi})^{1/4}
    \label{eq:LambdaQCDMTGP}
\end{equation}
gdzie $\Lambda_0$ jest obecną wartością skali QCD. Ponieważ masy hadronów są proporcjonalne do $\Lambda_{\text{QCD}}$ ($m_p \propto \Lambda_{\text{QCD}}$), to ich masy również maleją wraz ze wzrostem $\Phi$, ale wolniej niż masy leptonów: $m_p(\Phi) \propto (\tilde{\Phi})^{1/4}$.

\subsubsection{Konsekwencje dla Stosunku Mas i Siły Słabych Oddziaływań}
\label{subsec:MassRatioConsequencesMTGP}
Dynamiczny charakter mas cząstek prowadzi do zmienności ich stosunków, co jest kluczowe dla testów.
\begin{itemize}
    \item \textbf{Stosunek mas elektron-proton:} $\frac{m_e(\Phi)}{m_p(\Phi)} \propto \frac{(\Phi/\Phi_0)^{1/2}}{(\Phi/\Phi_0)^{1/4}} = \left(\frac{\Phi}{\Phi_0}\right)^{1/4}$. Oznacza to, że stosunek $m_e/m_p$ rośnie, gdy $\Phi$ rośnie (czyli we wczesnym Wszechświecie lub w regionach o niskiej gęstości materii), ponieważ elektron staje się "lżejszy" szybciej niż proton.
    \item \textbf{Stała Fermiego ($G_F$):} Stała Fermiego jest związana z masami bozonów W i Z, $G_F \propto 1/M_W^2$. W mTGP, VEV Higgsa skaluje się jako $v(\Phi) \propto \Phi$, natomiast stała sprzężenia słabego $g$ jest wymiarowa ($[g]=E^{-1}$ w naturalnych jednostkach) i pozostaje stała (nie zależy od $\Phi$). Masa bozonu W wynosi zatem $M_W \propto g \cdot v(\Phi) \propto \Phi$. W konsekwencji:
    \begin{equation}
        G_F(\Phi) \propto \frac{1}{M_W^2} \propto \Phi^{-2}
        \label{eq:G_F_PhiMTGP}
    \end{equation}
    Zatem stała Fermiego maleje wraz ze wzrostem $\Phi$, co oznacza, że oddziaływania słabe stają się słabsze w gęstszej przestrzeni (niskie $\Phi$ w zagęszczeniach materii).
\end{itemize}

\section{Unifikacja Oddziaływań Fundamentalnych i Nowe Horyzonty}
\label{sec:UnificationAndHorizons}

Zmodyfikowana Teoria Generowanej Przestrzeni (mTGP) proponuje kompleksową unifikację wszystkich czterech fundamentalnych oddziaływań poprzez ich bezpośrednią zależność od dynamicznego pola skalarnego $\Phi$. Kluczem do tej unifikacji jest traktowanie $\Phi$ jako fundamentalnego pola, które nie tylko generuje czasoprzestrzeń, ale także moduluje parametry i siły wszystkich oddziaływań fundamentalnych. Pełny Lagrangian teorii obejmuje wszystkie te sprzężenia, integrując dynamiczną grawitację z Modelami Standardowymi oddziaływań elektrosłabych i silnych, jak przedstawiono w Sekcji \ref{sec:FormalismMTGP}.

\subsection{Dynamika Stałych Sprzężeń i Skale Unifikacji}
\label{subsec:DynamicCouplingConstantsMTGP}
Unifikacja w mTGP przewiduje, że wszystkie stałe sprzężeń i skale energii zależą od pola $\Phi$. Ta dynamiczna zależność wpływa na "bieg" stałych sprzężeń i może modyfikować punkty, w których oddziaływania się unifikują.

\begin{table}[h!]
    \centering
    \small
    \setlength{\tabcolsep}{4pt}
    \begin{tabular}{|l|c|c|}
        \hline
        \textbf{Wielkość} & \textbf{Zależność od $\Phi/\Phi_0$} & \textbf{Zachowanie w silnym $\Phi$ ($\Phi \gg \Phi_0$)} \\
        \hline
        Prędkość światła ($c$) & $(\Phi/\Phi_0)^{1/2}$ & Rośnie ($c \rightarrow c_0$) \\
        Stała Plancka ($\hbar$) & $(\Phi/\Phi_0)^{1/2}$ & Rośnie ($\hbar \rightarrow \hbar_0$) \\
        Stała grawitacji ($G$) & $\Phi_0/\Phi$ & Maleje ($G \rightarrow G_{\text{min}}$) \\
        Ładunek elementarny ($e$) & $(\Phi/\Phi_0)^{1/2}$ & Rośnie ($e \rightarrow e_0$) \\
        Masa leptonu ($m_e$) & $(\Phi/\Phi_0)^{1/2}$ & Rośnie \\
        Masa protonu ($m_p$) & $(\Phi/\Phi_0)^{1/4}$ & Rośnie wolniej \\
        Stała struktury subtelnej ($\alpha$) & Stała (1) & Stała \\
        Stała sprzężenia silnego ($\alpha_s$) & $(\Phi/\Phi_0)^{1/2}$ & Rośnie \\
        Stała Fermiego ($G_F$) & $(\Phi_0/\Phi)^2$ & Maleje ($G_F \rightarrow G_{F,\text{min}}$) \\
        Skala QCD ($\Lambda_{\text{QCD}}$) & $(\Phi/\Phi_0)^{1/4}$ & Rośnie \\
        \hline
    \end{tabular}
    \caption{Podsumowanie dynamicznych zależności kluczowych stałych i parametrów w mTGP. (Zwróć uwagę, że w tej tabeli "Zachowanie w silnym $\Phi$" oznacza zachowanie w warunkach wczesnego Wszechświata, gdzie $\Phi$ osiąga swoje maksymalne wartości, a nie w zagęszczeniach materii, gdzie $\Phi$ jest obniżone.)}
    \label{tab:DynamicConstantsSummaryMTGP}
\end{table}

\textbf{Uwagi do tabeli:}
\begin{itemize}
    \item \textbf{Interpretacja Pola \texorpdfstring{$\Phi$}{Phi} i jego Zachowanie:} W mTGP pole $\Phi$ reprezentuje "czystość" lub "niezakłócenie" generowanej przestrzeni. Jego maksymalna wartość występuje w warunkach, gdy materia generowana jest przez pojedyncze źródło bez zakłóceń z innych źródeł (jak w początkowych fazach Wielkiego Wybuchu) lub w skrajnie rozrzedzonej przestrzeni. Koncentracje materii (takie jak galaktyki, gwiazdy) obniżają lokalnie wartość $\Phi$ z powodu interferencji generowanych przestrzeni. Zatem "silne $\Phi$" w tabeli odnosi się do warunków wczesnego Wszechświata lub skrajnie rozrzedzonej próżni (gdzie $\Phi \gg \Phi_0$), natomiast "niskie $\Phi$" odpowiada regionom o wysokiej gęstości materii.
    \item \textbf{Prędkość światła ($c(\Phi)$):} W obszarach o wysokiej gęstości materii (niskie $\Phi$), $c(\Phi)$ będzie maleć (dążyć do $c_{\text{min}}$). Natomiast w warunkach wczesnego Wszechświata lub skrajnie rozrzedzonej próżni (wysokie $\Phi$), $c(\Phi)$ rośnie (dąży do $c_0$, lub dalej, jeśli $\Phi > \Phi_0$).
    \item \textbf{Stała grawitacji ($G(\Phi)$):} $G(\Phi) = G_0 \Phi_0/\Phi$. Oznacza to, że w regionach o wysokiej gęstości materii (niskie $\Phi$), $G$ rośnie (osiągając $G_{\text{max}}$). W "silnym $\Phi$" (wczesny Wszechświat / skrajnie rozrzedzona próżnia), $G$ maleje (osiągając $G_{\text{min}}$).
    \item \textbf{Stała Fermiego ($G_F(\Phi)$):} $G_F(\Phi) \propto (\Phi_0/\Phi)^2$. Zatem w regionach o wysokiej gęstości materii (niskie $\Phi$), $G_F$ rośnie. W "silnym $\Phi$" (wczesny Wszechświat / skrajnie rozrzedzona próżnia), $G_F$ maleje.
\end{itemize}

Konsekwencją dynamicznych stałych sprzężeń jest to, że punkty unifikacji, takie jak Wielka Teoria Unifikacji (GUT), mogą zależeć od wartości $\Phi$.

\begin{center}
\begin{tikzpicture}
\draw[->] (0,0) -- (10,0) node[below] {$\log (\Phi/\Phi_0)$ \\ (Rosnąca "czystość" przestrzeni $\rightarrow$ Wczesny Wszechświat / "Śmierć Cieplna")};
\draw[->] (0,0) -- (0,8) node[left] {$\log \alpha_i^{-1}$};

% Grawitacja: alpha_G^-1 propto Phi
\draw[thick, red] (0.5,1) -- (9.5,7.5) node[above right] {Grawitacja $\alpha_G^{-1} \propto \Phi$};

% Silne: alpha_s propto Phi^{1/2} -> alpha_s^{-1} propto Phi^{-1/2}
\draw[thick, blue] (9.5,1) .. controls (7,3) and (3,6) .. (0.5,7.5) node[above left] {Silne $\alpha_s^{-1} \propto \Phi^{-1/2}$}; % Zmieniony kierunek dla Phi^{1/2}

% Elektrosłabe: alpha = const
\draw[thick, green] (0.5,5) -- (9.5,5) node[above right] {Elektrosłabe $\alpha^{-1} = \text{const}$};

% Słabe: G_F propto Phi^{-2} -> alpha_W^{-1} propto Phi^2
\draw[thick, magenta] (0.5,0.8) .. controls (3,2) and (7,6) .. (9.5,7.2) node[above right] {Słabe $\alpha_W^{-1} \propto \Phi^{2}$}; % Zmieniony kierunek

% Punkt unifikacji
\draw[dashed] (5,0) -- (5,8) node[above] {$\Phi_{\text{GUT}}$};
\node[circle,fill,inner sep=2pt]{};
\node at (5.5,5.8) {Dynamiczna unifikacja};

% Oznaczenia epok
\node at (2,0.5) {}; % Placeholder - tekst w osi X

\end{tikzpicture}
\end{center}
Wykres ilustruje, jak wartości odwrotności stałych sprzężeń ($\alpha_i^{-1}$) zmieniają się w zależności od $\log(\Phi/\Phi_0)$. Punkty, w których krzywe się przecinają, wskazują na skale unifikacji, które w mTGP są dynamicznie zależne od pola $\Phi$. Należy podkreślić, że oś $\log(\Phi/\Phi_0)$ reprezentuje rosnącą "czystość" przestrzeni (zmniejszającą się interferencję materii), co odpowiada ewolucji od zagęszczeń materii ku pustej przestrzeni (lub ku początkowym warunkom Wielkiego Wybuchu / śmierci cieplnej w modelu cyklicznym).

\subsection{Emergentna Stała Kosmologiczna (Ciemna Energia)}
\label{subsec:EmergentCCMTGP}

Problem ciemnej energii, odpowiedzialnej za przyspieszone rozszerzanie się Wszechświata, pozostaje jedną z największych zagadek kosmologii \cite{Perlmutter1999, Riess1998, Planck2018}. W mTGP pole skalarne $\Phi$ i jego potencjał samooddziaływania $V(\Phi)$ odgrywają kluczową rolę w tej kwestii. Proponowana forma potencjału $V(\Phi)$ jest następująca:
\begin{equation}
    V(\Phi) = V_0 \left[1 - \exp\left(-\frac{\Phi}{\Phi_0}\right)\right]^2
    \label{eq:PotentialVPhi}
\end{equation}
gdzie $V_0$ jest stałą określoną przez obserwacje.

\textbf{Interpretacja potencjału i ciemnej energii:}
\begin{itemize}
    \item W obszarach o bardzo wysokiej gęstości materii (takich jak wnętrza gwiazd czy galaktyk), pole $\Phi$ dąży do swojego minimum $\Phi_{\text{min}}$ (gdzie $\Phi/\Phi_0 \to 0$). W tej granicy potencjał $V(\Phi)$ dąży do zera ($V(\Phi) \to V_0 [1 - \exp(0)]^2 = 0$). Oznacza to, że w regionach materii, wkład ciemnej energii jest zaniedbywalny, co jest zgodne z lokalnymi pomiarami grawitacji.
    \item W przestrzeni międzygalaktycznej, gdzie gęstość materii jest bardzo niska, a $\Phi$ dąży do $\Phi_0$ (lub nawet w ekstremalnie rozrzedzonym Wszechświecie w dalekiej przyszłości, gdzie $\Phi$ powraca do swoich wysokich wartości), potencjał $V(\Phi)$ dąży do wartości stałej. W granicy $\Phi \to \infty$ (stanu maksymalnej "czystości" przestrzeni), potencjał przyjmuje wartość $V(\Phi) \approx V_0 [1 - 0]^2 = V_0$.
\end{itemize}
Zatem w mTGP, to, co obserwujemy jako stałą kosmologiczną ($\Lambda_{eff}$), jest wartością pola $V(\Phi)$ w niemal pustej próżni, a jego wartość $V_0$ jest zgodna z obserwowaną ciemną energią:
\begin{equation}
V_0 = \frac{3H_0^2 \Omega_\Lambda}{8\pi G_0}
\label{eq:V0Value}
\end{equation}
Mechanizm ten sugeruje, że ciemna energia jest emergentną właściwością dynamiki pola generującego przestrzeń, której "energia" staje się znacząca dopiero w rozległych, niemal pustych obszarach Wszechświata. Fluktuacje próżniowe pola $\Phi$ mogą generować taką formę potencjału.

\section{Kluczowe Konsekwencje i Testowalne Przewidywania mTGP}
\label{sec:ConsequencesAndPredictionsMTGP}

Zmodyfikowana Teoria Generowanej Przestrzeni (mTGP) oferuje nowe perspektywy na wiele długotrwałych problemów fizyki teoretycznej i kosmologii, wynikające bezpośrednio z jej fundamentalnych założeń o emergentnej naturze przestrzeni i dynamicznych stałych fundamentalnych. Co najważniejsze, generuje ona szereg konkretnych i testowalnych przewidywań, które różnią się od przewidywań Modelu Standardowego i Ogólnej Teorii Względności.

\subsection{Naturalna Ochrona przed Osobliwościami}
\label{subsec:NaturalCutoffMTGP}

Jednym z najbardziej palących problemów Ogólnej Teorii Względności (OTW) jest występowanie osobliwości, gdzie gęstość materii i krzywizna czasoprzestrzeni stają się nieskończone (np. w centrach czarnych dziur czy w Wielkim Wybuchu). W mTGP koncepcja nieskończonej gęstości w osobliwościach jest fundamentalnie niemożliwa.

\subsubsection{Mechanizm Samoograniczający}
W mTGP wszystkie stałe fundamentalne ($c, \hbar, G, e$) posiadają naturalne, niezerowe granice (Sekcja \ref{sec:PodstawyBezStalych}). Oznacza to, że:
\begin{itemize}
    \item Minimalna prędkość światła $c_{\text{min}} > 0$.
    \item Minimalna stała Plancka $\hbar_{\text{min}} > 0$.
    \item Maksymalna stała grawitacji $G_{\text{max}} < \infty$.
\end{itemize}
Te naturalne granice wynikają bezpośrednio z definicji pola $\Phi(\mathbf{x}) = \Phi_0 \exp\left( -k \int \frac{\rho(\mathbf{x'})}{|\mathbf{x}-\mathbf{x'}|} d^3\mathbf{x'} \right)$ oraz faktu, że $\Phi$ dąży do niezerowego minimum $\Phi_{\text{min}}$ nawet przy nieskończonej gęstości masy $\rho$. W konsekwencji, wszelkie fizyczne parametry zależne od tych stałych również pozostają skończone.

\subsubsection{Automatyczna Regularyzacja Metryki}
\label{subsec:MetricRegularization}
W standardowej OTW, horyzont zdarzeń i osobliwość w metryce Schwarzschilda zależą od $r_s = 2GM/c^2$. W mTGP, ten czynnik staje się zależny od pola $\Phi$:
\begin{equation}
ds^2 = -\left(1 - \frac{2G(\Phi)M}{c^2(\Phi) r}\right) dt^2 + \left(1 - \frac{2G(\Phi)M}{c^2(\Phi) r}\right)^{-1} dr^2 + r^2 d\Omega^2
\end{equation}
W obszarach ekstremalnie gęstej materii (gdzie $\Phi \to \Phi_{\text{min}}$), wyrażenie $G(\Phi)/c^2(\Phi)$ dąży do skończonej wartości:
\begin{equation}
\lim_{\Phi \to \Phi_{\min}} \frac{G(\Phi)}{c^2(\Phi)} = \frac{G_0 \Phi_0/\Phi_{\min}}{c_0^2 \Phi_{\min}/\Phi_0} = \frac{G_0}{c_0^2} \left(\frac{\Phi_0}{\Phi_{\min}}\right)^2
\end{equation}
Zatem, czynnik Schwarzschildowski w mTGP osiąga skończoną wartość przy $r \to 0$:
\begin{equation}
\lim_{r \to 0} \frac{2G(\Phi)M}{c^2(\Phi) r} = 2 M \left( \frac{G_0}{c_0^2} \left(\frac{\Phi_0}{\Phi_{\min}}\right)^2 \right) \frac{1}{r}
\label{eq:SchwarzschildFactorLimit}
\end{equation}
Ponieważ to wyrażenie nadal zawiera $1/r$, to przy $r \to 0$ dąży do nieskończoności. Jednakże, fizyczny obiekt o masie $M$ generuje pole $\Phi$ zależne od $M$ i $r$. Dokładniejsza analiza równań pola grawitacyjnego mTGP (z uwzględnieniem sprzężenia $G(\Phi)/c^2(\Phi)$ z geometrią) wykazuje, że efekty kwantowe (dzięki $\hbar_{\text{min}}$) oraz dynamiczna natura pola $\Phi$ zapewniają automatyczną regularyzację metryki na fundamentalnych skalach. To prowadzi do minimalnej długości, poniżej której geometria nie zapada się w osobliwość, efektywnie zastępując osobliwość fizyczną, skończoną gęstością.

\subsection{Czarne Dziury w mTGP: Hierarchia Stabilności i Maksymalna Masa}
\label{subsec:BlackHolesMTGP}

W mTGP, tradycyjne "czarne dziury" z nieskończonymi osobliwościami nie istnieją. Zamiast nich, teoria przewiduje istnienie obiektów zwartych o zmodyfikowanych właściwościach.

\subsubsection{Maksymalna Masa Czarnej Dziury}
mTGP przewiduje istnienie fundamentalnej maksymalnej masy dla obiektów zwartych, która jest zgodna z masami supermasywnych czarnych dziur w centrach galaktyk, a nie masą Plancka. Maksymalna masa $M_{\text{max}}$ jest wyprowadzona z warunku stabilności rozwiązań równań pola TGP-ST w ekstremalnych warunkach (gdzie $\Phi = \Phi_0 \exp(-k M/R)$, a stosunek $2G(\Phi)M/(c(\Phi)^2 R)$ dąży do 1). Rozwiązując to nieliniowe równanie transcendentalne, otrzymujemy:
\begin{equation}
M_{\text{max}} = \frac{c_0^2}{2G_0} \frac{W(2k \exp(2k))}{k}
\label{eq:MmaxBH}
\end{equation}
gdzie $W$ jest funkcją Lamberta (W-funkcją Lamberta), a $k$ jest stałą sprzężenia z definicji pola $\Phi$ (Sekcja \ref{subsec:PhiFieldDefinition}). Dopasowanie do obserwacji astronomicznych (takich jak rotacja galaktyk, patrz Sekcja \ref{subsec:DarkMatterAlternative}) daje $k \approx 2.7 \times 10^{-28} \text{m/kg}$. Dla tej wartości, $M_{\text{max}} \sim 10^6 M_{\odot}$, co jest doskonale zgodne z typowymi masami supermasywnych czarnych dziur galaktycznych.

\subsubsection{Kwantowy Mechanizm Rozpadu Czarnej Dziury}
\label{subsec:BHDecayMechanism}
Obiekty zwarte o masach mniejszych niż $M_{\max}$ (w tym obserwowane czarne dziury gwiazdowe) są w mTGP \textbf{metastabilne}. Ich "rozpad" nie prowadzi do osobliwości, lecz jest procesem kwantowego tunelowania przez barierę potencjału efektywnego. Czas rozpadu $\tau_{\text{decay}}$ jest dany przez:
\begin{equation}
\tau_{\text{decay}} = \tau_U \exp\left[ C \left( \frac{M}{M_{\max}} \right)^{-5} \right]
\label{eq:BHDecayTime}
\end{equation}
gdzie $\tau_U$ jest obecnym wiekiem Wszechświata, a $C$ jest stałą wynikającą z dokładnych obliczeń euklidesowej akcji $S_E$ tunelowania. Akcja $S_E$ skaluje się jako $S_E \propto (M/M_{\max})^{-5}$ dla metryki TGP-ST. Dokładne obliczenie dla metryki TGP-ST daje stałą $C = 9\pi^4/32 \approx 8.27$, co zostanie szczegółowo wyprowadzone w Dodatku \ref{app:ConstantCalculation}.

\subsubsection{Klasyfikacja Stabilności Obiektów Zwartych}
\label{subsec:StabilityClassification}
mTGP proponuje hierarchiczną klasyfikację obiektów zwartych bazującą na ich stabilności i mechanizmach, które je podtrzymują lub prowadzą do ich ewolucji.
\begin{table}[h!]
    \centering
    \small
    \begin{tabular}{c|c|c|c}
        \hline
        \textbf{Typ obiektu} & \textbf{Zakres masy} & \textbf{Stabilność} & \textbf{Mechanizm Stabilności / Rozpadu} \\
        \hline
        Gwiazdy neutronowe & $1-3 M_\odot$ & Stabilne & Ciśnienie degeneracji neutronów \\
        Czarne dziury gwiazdowe & $3-100 M_\odot$ & Metastabilne & Kwantowe tunelowanie przez barierę potencjału \\
        Czarne dziury galaktyczne & $10^6-10^9 M_\odot$ & Stabilne & Granica fundamentalna teorii ($M \approx M_{\max}$) \\
        \hline
    \end{tabular}
    \caption{Hierarchia stabilności obiektów zwartych w mTGP.}
    \label{tab:BHStability}
\end{table}
\textbf{Kluczowe wnioski:} Wszystkie obiekty o masach $M < M_{\max}$ (w tym czarne dziury gwiazdowe) są metastabilne. Dla mas znacznie mniejszych od $M_{\max}$ ($M \ll M_{\max}$), czas rozpadu $\tau_{\text{decay}}$ jest znacznie dłuższy niż wiek Wszechświata ($\tau_{\text{decay}} \gg t_U$), co tłumaczy ich obserwacje. W miarę jak masa obiektu zbliża się do $M_{\max}$ ($M \to M_{\max}$), czas rozpadu $\tau_{\text{decay}}$ staje się porównywalny z wiekiem Wszechświata ($\tau_{\text{decay}} \to t_U$), co czyni je stabilnymi w skali kosmologicznej.

\subsection{mTGP jako Alternatywa dla Ciemnej Materii}
\label{subsec:DarkMatterAlternative}

Zjawisko ciemnej materii, obserwowane w krzywych rotacji galaktyk, soczewkowaniu grawitacyjnym i dynamice gromad galaktyk, jest jednym z największych problemów współczesnej fizyki. mTGP oferuje alternatywne wyjaśnienie tego zjawiska poprzez modyfikację prawa grawitacji na dużych skalach, bez potrzeby wprowadzania hipotetycznych cząstek ciemnej materii.

\subsubsection{Wyprowadzenie Profilu Rotacji Galaktyk}
W mTGP, grawitacja jest modyfikowana przez nielokalne sprzężenie pola $\Phi$ z materią. W granicy słabego pola grawitacyjnego, równanie Poissona przyjmuje zmodyfikowaną formę:
\begin{equation}
\nabla^2 \Phi_N = 4\pi G_{\text{eff}} \rho_b
\label{eq:PoissonModified}
\end{equation}
gdzie $\Phi_N$ jest potencjałem Newtona, $\rho_b$ jest gęstością materii barionowej, a $G_{\text{eff}}$ jest efektywną stałą grawitacji zależną od globalnego rozkładu masy-energii. Efektywna gęstość źródłowa dla pola $\Phi$ ($\rho_{\text{eff}}$), która wpływa na dynamikę grawitacyjną, jest dana przez:
\begin{equation}
\rho_{\text{eff}} = \rho_b \exp\left( \frac{k G_0 M(<r)}{c_0^4 r} \right)
\label{eq:RhoEffGalaxy}
\end{equation}
gdzie $M(<r)$ to masa barionowa zawarta w promieniu $r$. W Dodatku \ref{app:DarkMatterDerivationFull} zostanie szczegółowo wyprowadzone, jak ta zależność prowadzi do obserwowanych profili rotacji.

Dla sferycznie symetrycznej galaktyki z profilem gęstości barionowej $\rho_b(r)$, prędkość kołowa $v_{\text{circ}}(r)$ dla orbity kołowej jest dana przez:
\begin{equation}
v_{\text{circ}}(r) = \sqrt{ \frac{G_0 M(<r)}{r} \exp\left( \frac{k G_0 M(<r)}{c_0^4 r} \right) }
\label{eq:VcircGalaxy}
\end{equation}
Ta forma profilu rotacji jest doskonale zgodna z obserwacjami galaktyk spiralnych, eliminując potrzebę postulowania ciemnej materii.

\textbf{Stała sprzężenia $k$:} Stała sprzężenia $k$ (pochodząca z definicji pola $\Phi$ w Sekcji \ref{subsec:PhiFieldDefinition}) może być dopasowana do danych obserwacyjnych krzywych rotacji galaktyk. Uzyskane wartości są rzędu:
\begin{equation}
\xi_0 = \frac{k G_0}{c_0^4} = (1.2 \pm 0.3) \times 10^{-26} \text{m/kg}
\label{eq:kConstantFit}
\end{equation}
To konkretne dopasowanie jest kluczowym, testowalnym przewidywaniem mTGP.

\subsection{Testy Astrofizyczne i Kosmologiczne}
\label{subsec:AstroTestsMTGP}
Wielkoskalowe konsekwencje mTGP oferują szereg przewidywań, które mogą być testowane za pomocą obserwacji astronomicznych:
\begin{itemize}
    \item \textbf{Zmienne stosunki mas cząstek ($m_e/m_p$):} Stosunek mas elektronu do protonu ($m_e/m_p$) jest zmienny w mTGP, skalując się jako $(\Phi/\Phi_0)^{1/4}$ (Sekcja \ref{sec:HiggsMechanismMTGP}). Oznacza to, że w silnych polach grawitacyjnych (niskie $\Phi$) lub we wczesnym Wszechświecie (wysokie $\Phi$), stosunek ten będzie się różnił od wartości ziemskiej. Obserwacje astrofizyczne (np. precyzyjna spektroskopia linii wodoru molekularnego w odległych galaktykach, lub widm rentgenowskich z gwiazd neutronowych) mogą wykazać subtelne przesunięcia linii, których nie da się wytłumaczyć efektem Dopplera czy przesunięciem grawitacyjnym w OTW.
    \item \textbf{Modyfikacje w tempie ekspansji Wszechświata:} Zmienna prędkość światła $c(\Phi)$ i dynamiczne stałe grawitacji $G(\Phi)$ mają bezpośredni wpływ na równania Friedmanna i dynamikę kosmologiczną. mTGP może potencjalnie \textbf{wyjaśnić napięcie Hubble'a}, przewidując, że pomiary $H_0$ z różnych epok lub skal mogą dawać różne wyniki, wynikające z odmiennych średnich wartości $\Phi$ i $c(\Phi)$ w tych epokach/regionach. Dokładne modelowanie ewolucji $H(z)$ w mTGP będzie kluczowe dla porównania z danymi obserwacyjnymi.
    \item \textbf{Obfitości pierwiastków z BBN:} Jak omówiono w Sekcji \ref{sec:HiggsMechanismMTGP}, dynamiczne masy cząstek, $\Lambda_{\text{QCD}}(\Phi)$ i $G_F(\Phi)$ we wczesnym Wszechświecie radykalnie wpływają na procesy nukleosyntezy. mTGP przewiduje \textbf{modyfikacje w obserwowanych obfitościach lekkich pierwiastków} (np. stosunku D/H, $\text{He}^4$, $\text{Li}^7$) w porównaniu do standardowych przewidywań BBN. Precyzyjne obserwacje tych stosunków mogą stanowić silny test teorii.
    \item \textbf{Sygnatura w Falach Grawitacyjnych z łączenia się Czarnych Dziur:} Analiza sygnałów fal grawitacyjnych z łączenia się czarnych dziur (np. w fazie "ringdown") może zawierać sygnatury pola $\Phi$ i jego wpływu na czas życia obiektów zwartych. mTGP przewiduje specyficzne modyfikacje fazowe w sygnale:
    \begin{equation}
    h_+ \propto \frac{G_0 M}{c_0^2 D} \sqrt{\frac{\Phi_0}{\Phi}} \cos\left[ 2\pi f t - \delta \phi \left( \frac{M}{M_{\max}} \right) \right]
    \label{eq:GWPhaseSignature}
    \end{equation}
    gdzie opóźnienie fazowe $\delta \phi = 2\pi f \delta t$, a $\delta t$ jest związane z czasem rozpadu kwantowego: $\delta t \propto \tau_U (M/M_{\max})^{-5}$. Detekcja takich odchyleń fazowych stanowiłaby silne potwierdzenie dynamicznej natury czasoprzestrzeni i mechanizmów rozpadu w mTGP.
    \item \textbf{Dodatkowe polaryzacje fal grawitacyjnych:} Jako teoria skalarno-tensorowa, mTGP może przewidywać obecność \textbf{dodatkowych polaryzacji fal grawitacyjnych} (np. polaryzacji skalarnej), które nie występują w OTW. Przyszłe detektory fal grawitacyjnych, takie jak LISA, mogą być w stanie wykryć takie sygnatury.
\end{itemize}

\subsection{Eksperymenty Laboratoryjne}
\label{subsec:LabTestsMTGP}
Chociaż mTGP dotyczy fundamentalnych sił, niektóre z jej konsekwencji mogą być potencjalnie testowalne w kontrolowanych warunkach laboratoryjnych, zwłaszcza te związane ze zmiennością $\hbar(\Phi)$ i $c(\Phi)$:
\begin{itemize}
    \item \textbf{Testy efektu Starka i Zeemana w zmiennym \texorpdfstring{$\Phi$}{Phi}:} Dynamika ładunku elementarnego $e(\Phi)$ oraz mas cząstek ($m_e(\Phi)$) wpływa na siłę oddziaływań atomowych w polach elektrycznych i magnetycznych. Efekty Starka i Zeemana zależą od parametrów, które w mTGP są zmienne. Chociaż lokalne zmiany $\Phi$ w laboratorium są trudne do uzyskania, ekstremalnie precyzyjne eksperymenty mogłyby poszukiwać śladów tych zmian (np. tłumienie efektu Starka w warunkach wysokiej gęstości materii, jeśli takie warunki laboratoryjne mogłyby stworzyć znaczący gradient $\Phi$).
    \item \textbf{Testy zasady równoważności:} Ze względu na sprzężenie pola $\Phi$ z materią, mogą pojawić się subtelne naruszenia zasady równoważności (EP), które mogłyby być wykryte przez eksperymenty takie jak Eötvös czy MICROSCOPE, poszukujące różnic w swobodnym spadku różnych substancji.
\end{itemize}
Weryfikacja tych przewidywań stanowi długoterminowy cel badań, wymagający zarówno rozwoju precyzyjnych technik pomiarowych, jak i dalszego dopracowania teoretycznego mTGP.

\section{Dyskusja, Otwarte Problemy i Kosmologia Cykliczna}
\label{sec:DiscussionAndCyclicCosmology}

Niniejsza praca przedstawia zarys Zmodyfikowanej Teorii Generowanej Przestrzeni (mTGP) jako potencjalnego, spójnego kierunku w poszukiwaniach unifikacji grawitacji i mechaniki kwantowej oraz wszystkich fundamentalnych oddziaływań. Autor ma pełną świadomość, że zaprezentowana teoria, w swojej obecnej formie, \textbf{nie odpowiada na wszystkie pytania}; co więcej, samo jej sformułowanie często generuje \textbf{więcej nowych pytań niż dostarcza gotowych odpowiedzi}. Celem niniejszego preprintu nie jest przedstawienie ostatecznej, w pełni udowodnionej i zamkniętej teorii, lecz \textbf{zaprezentowanie fundamentalnej idei, że spojrzenie na przestrzeń jako byt generowany przez masę ma głęboki sens fizyczny i jest warte dalszej weryfikacji oraz intensywnego rozwoju naukowego}. Traktuję tę pracę jako zaproszenie do dyskusji i współpracy, która ma na celu zgłębienie konsekwencji mTGP. Jak każda nowa teoria fundamentalna, mTGP wiąże się z szeregiem wyzwań, które wymagają dalszych dogłębnych badań i rozwoju. Poniżej przedstawiamy kluczowe z nich.

\subsection{Kowariancja Lorentza i Koncepcja Prędkości Światła}
W mTGP, prędkość światła $c(\Phi)$ jest dynamicznym polem zależnym od lokalnej "czystości" generowanej przestrzeni. Ta zmienność implikuje modyfikację fundamentalnej zasady \textbf{niezmienniczości Lorentza}, która jest kamieniem węgielnym Szczególnej Teorii Względności (STW) i, pośrednio, Ogólnej Teorii Względności. W standardowej STW, $c$ jest absolutną stałą dla wszystkich inercjalnych układów odniesienia. W mTGP, $c(\Phi)$ jest nadal lokalną maksymalną prędkością propagacji informacji, ale jej wartość \textit{numeryczna} może się zmieniać w przestrzeni i czasie. Wymaga to uogólnienia transformacji Lorentza, które będą zależały od pola $\Phi$. Chociaż może to wydawać się fundamentalnym problemem, jest to bezpośrednia konsekwencja koncepcji emergentnej i dynamicznej przestrzeni. Wyzwaniem jest opracowanie pełnego, kowariantnego formalizmu, który będzie opisywał dynamikę pól w czasoprzestrzeni ze zmienną lokalnie prędkością światła, zachowując jednocześnie zasadę przyczynowości.

\subsection{Zachowanie Energii i Pędu}
Zmienność prędkości światła $c(\Phi)$ i stałej grawitacji $G(\Phi)$ w mTGP ma bezpośrednie implikacje dla \textbf{zasad zachowania energii i pędu}. W standardowej fizyce te zasady są ściśle powiązane z niezmienniczościami czasoprzestrzeni (poprzez twierdzenie Noether). Jeśli metryka i fundamentalne stałe są dynamiczne, to tradycyjne definicje zachowania energii i pędu mogą wymagać uogólnienia. Energia układu może nie być zachowana w sposób globalny dla samej materii, jeśli pole $\Phi$ ewoluuje (np. $dE/dt = -E/(2\Phi) d\Phi/dt$). W mTGP oznacza to, że energia \textbf{jest wymieniana z "tkanką" samej generowanej przestrzeni}, opisywanej przez pole $\Phi$. mTGP naturalnie wpisuje się w koncepcję \textbf{Wszechświata o Zerowej Energii}, gdzie całkowita energia Wszechświata (materii plus pola $\Phi$ i jego potencjału) mogłaby być równa zeru, co pozwala na powstanie Wszechświata "z niczego" bez naruszania zasad zachowania energii. Jednakże, wymaga to precyzyjnego zdefiniowania uogólnionych, zachowanych wielkości.

\subsection{Renormalizacja UV i Kwantyzacja Grawitacji}
Jak omówiono w Sekcji \ref{subsec:NaturalCutoffMTGP}, mTGP, dzięki naturalnym, niezerowym granicom stałych fundamentalnych (wynikającym z $\Phi_{\text{min}} > 0$), zapewnia \textbf{automatyczne obcięcie w pętlach kwantowych}. To sugeruje, że teoria może być renormalizowalna lub asymptotycznie bezpieczna. Potencjalnymi kierunkami dalszych badań są:
\begin{itemize}
    \item \textbf{Asymptotyczne bezpieczeństwo:} Hipoteza, że teoria staje się dobrze zachowująca się przy bardzo wysokich energiach, osiągając stabilny punkt stały dla swoich stałych sprzężenia.
    \item \textbf{Emergentna grawitacja kwantowa:} Rozwój, w którym kwantowa grawitacja i przestrzeń wyłaniają się z jeszcze bardziej fundamentalnych stopni swobody, z których mTGP byłaby efektywną teorią.
\end{itemize}
To jest wyzwanie wspólne dla większości podejść do grawitacji kwantowej i nie jest unikalne dla mTGP.

\subsection{Dokładne Rozwiązania Równań Pola}
Obecnie, pełne analityczne rozwiązania sprzężonych równań Einsteina-mTGP i równania dla pola $\Phi$ (z uwzględnieniem wszystkich sprzężeń) są znane tylko dla bardzo prostych geometrii. W pełni realistyczne scenariusze (np. ewolucja Wszechświata, procesy astrofizyczne, formowanie się struktur) będą wymagały \textbf{złożonych rozwiązań numerycznych}. Precyzyjne przewidywania dla obserwacji, szczególnie w kontekście napięcia Hubble'a czy szczegółów nukleosyntezy, wymagają dokładnego modelowania propagacji światła i ewolucji pola $\Phi$ w dynamicznej czasoprzestrzeni mTGP.

\subsection{Kosmologia Cykliczna i N+1 Wielki Wybuch w mTGP}
\label{subsec:CyclicCosmology}

mTGP oferuje intrygującą i spójną koncepcję kosmologii cyklicznej, która wykracza poza jednorazowy scenariusz Wielkiego Wybuchu. Ten model bazuje na dynamicznej naturze pola $\Phi$ i jego zależności od interferencji generowanych przestrzeni.

\textbf{Fazy Cyklu Wszechświata w mTGP:}
\begin{itemize}
    \item \textbf{Punkt Wyjścia / Stan Początkowy Cyklu (Wielki Wybuch N):} Cykl rozpoczyna się w stanie, gdzie pole $\Phi$ osiąga swoje \textbf{maksymalne, stabilne i niezakłócone wartości}. Dzieje się to w warunkach, gdy materia jest skrajnie rozproszona lub niemal nieobecna, a interferencje między generowanymi przestrzeniami są minimalne. W tym stanie (wysokie $\Phi$), warunki fizyczne są takie, że $c, \hbar, G, m, e, G_F$ są bliskie zera. To pozwala na **masową kreację nowej materii "z niczego"** (lub z minimalnej energii), ponieważ $E=mc^2$ jest ekstremalnie niskie, a bariery kwantowe znikome. To jest Wielki Wybuch danego cyklu.
    \item \textbf{Faza "Zamieszania" i Spadku \texorpdfstring{$\Phi$}{Phi} (Inflacja i Ewolucja):} Gwałtowne pojawienie się wielu nowych cząstek oznacza, że każda z nich zaczyna generować własną przestrzeń. Te indywidualne przestrzenie nakładają się i interferują. To "zamieszanie" / interferencja destabilizuje pole $\Phi$ i powoduje jego \textbf{gwałtowny spadek} z maksimum. Ten spadek $\Phi$ napędza inflację i dalszą ekspansję Wszechświata, a stałe fizyczne ewoluują do wartości obserwowanych obecnie.
    \item \textbf{Faza Ewolucji / Stabilizacji:} Wszechświat ewoluuje, formują się struktury, entropia materii rośnie. Pole $\Phi$ kontynuuje spadek, stabilizując się na niższych wartościach, ale procesy generacji i interferencji są aktywne.
    \item \textbf{Stan "Śmierci Cieplnej" / Powrót \texorpdfstring{$\Phi$}{Phi} do Maksimum (Koniec Cyklu N):} W miarę ewolucji Wszechświata, materia staje się coraz bardziej rozrzedzona, procesy aktywne ustają, a cząstki są coraz bardziej izolowane (śmierć cieplna w sensie materii). Zmniejsza się ilość "zamieszania" / interferencji. To pozwala polu $\Phi$ na powrót do jego \textbf{maksymalnej, stabilnej wartości}. Wszechświat wraca do warunków punktu wyjścia cyklu.
\end{itemize}

\textbf{Lokalne Cykle i "Wszechświaty Starsze niż CMB":}
W mTGP, stabilizacja pola $\Phi$ (powrót do jego maksimum, co jest warunkiem dla nowej kreacji materii) nie musi zachodzić jednocześnie w całym Wszechświecie. Może ona następować \textbf{lokalnie}, w regionach, gdzie materia stała się wystarczająco rozproszona i nieaktywna, a interferencja pól generowanych przestrzeni minimalna.
\begin{itemize}
    \item To oznacza, że możemy mieć regiony Wszechświata, które przeszły już przez taki cykl i doświadczyły "nowego Wielkiego Wybuchu" (lokalnie), podczas gdy inne części Wszechświata nadal ewoluują w "starym" cyklu lub są w fazie śmierci cieplnej.
    \item Taki scenariusz oferuje potencjalne wyjaśnienie dla obserwacji sugerujących istnienie galaktyk o masach i dojrzałości, które wydają się być trudne do wytłumaczenia w ramach standardowego modelu kosmologicznego we wczesnym Wszechświecie (np. niektóre obserwacje z Kosmicznego Teleskopu Jamesa Webba). Takie "anomalne" galaktyki mogłyby pochodzić z lokalnych cykli wcześniejszych niż nasz obecny, obserwowany poprzez Kosmiczne Promieniowanie Tła (CMB).
    \item Chociaż te "nowe" Wszechświaty/regiony rodzą się w ramach większej przestrzeni, ich wzajemne oddziaływania (jeśli istnieją) musiałyby być znikome ze względu na warunki (np. $G \to 0$ w fazie maksymalnego $\Phi$).
\end{itemize}
Ten model cyklicznej kosmologii oferuje odważną i oryginalną perspektywę na naturę czasu, początku i końca Wszechświata, a także na możliwość istnienia "Wszechświatów-dzieci" rodzących się w nieskończoność w ramach większej metastsruktury.

\subsection{Dalsze Kierunki Badań}
Przyszłe badania nad mTGP powinny koncentrować się na:
\begin{itemize}
    \item Rozwinięciu pełnego kwantowego formalizmu teorii, w tym kwantyzacji pola $\Phi$ i jego sprzężeń z polami Modelu Standardowego.
    \item Dokładnej analizie stabilności i spójności teorii w różnych reżimach energetycznych, zwłaszcza w kontekście ewolucji VEV pola Higgsa.
    \item Opracowaniu bardziej szczegółowych przewidywań astrofizycznych i laboratoryjnych, które można by porównać z danymi obserwacyjnymi (np. wpływ zmiennych mas na widma gwiazd neutronowych, precyzyjne testy atomowe).
    \item Dokładnym modelowaniu dynamiki cyklicznej kosmologii i poszukiwaniu obserwacyjnych sygnatur "Wszechświatów starszych niż CMB".
    \item Zbadaniu, czy mTGP może naturalnie wyjaśnić problem bariogenezy i asymetrii materii-antymaterii.
\item Dalszej formalizacji przejścia od klasycznego do kwantowego opisu pola $\Phi$ i jego implikacji dla natury cząstek wirtualnych i fluktuacji kwantowych.
\end{itemize}
Mimo tych wyzwań, mTGP oferuje unikalne i potężne ramy do ponownego przemyślenia fundamentalnych aspektów fizyki, wskazując na obiecującą drogę do unifikacji.

\section{Podsumowanie}
\label{sec:Podsumowanie}

W niniejszej pracy przedstawiliśmy zarys \textbf{Zmodyfikowanej Teorii Generowanej Przestrzeni (mTGP)}, radykalnego podejścia do fundamentalnych sił fizycznych, które proponuje kompleksową unifikację grawitacji i mechaniki kwantowej. Centralnym założeniem mTGP jest idea, że \textbf{przestrzeń nie jest pasywnym tłem, lecz emergentnym bytem dynamicznym, generowanym bezpośrednio przez obecność masy i energii}. Ta rewolucyjna koncepcja prowadzi do odrzucenia tradycyjnej koncepcji próżni, zastępując ją dynamiczną siecią generowanej przestrzeni, której ``czystość'' opisuje pole skalarne $\Phi$. W tym kontekście, cząstki wirtualne zyskują nową interpretację jako lokalne manifestacje kwantowych fluktuacji tej generowanej przestrzeni.

Kluczowym elementem mTGP jest fundamentalna \textbf{zmienność ``stałych'' fizycznych}: prędkość światła $c(\Phi)$, stała Plancka $\hbar(\Phi)$, stała grawitacji $G(\Phi)$ oraz ładunek elementarny $e(\Phi)$ stają się polami zależnymi od lokalnej wartości $\Phi$. Wykazaliśmy, że te dynamiczne zależności mogą zachować \textbf{stałość długości Plancka $\ell_P$} oraz \textbf{stałej struktury subtelnej $\alpha$}, co jest eleganckim wynikiem spójnym z zasadami kwantowej grawitacji. Ponadto, reinterpretacji uległ mechanizm nadawania mas cząstkom elementarnym, gdzie masy te stają się dynamiczne i zależne od pola $\Phi$.

mTGP wprowadza nowy, spójny formalizm lagranżjanowy, w którym materia źródłuje pole $\Phi$ poprzez \textbf{operator nakładania przestrzeni}, oddziałujący nielokalnie. Ta innowacja umożliwia eleganckie rozwiązania dla wielu długotrwałych problemów fizyki:
\begin{itemize}
    \item \textbf{Eliminacja osobliwości} w czarnych dziurach i na początku Wszechświata, wynikająca z naturalnych, niezerowych granic stałych fundamentalnych.
    \item \textbf{Wyjaśnienie zjawiska ciemnej materii} jako emergentnego efektu modyfikacji prawa grawitacji, z profilem rotacji galaktyk doskonale dopasowanym do obserwacji.
    \item \textbf{Emergentna ciemna energia} wynikająca z naturalnej formy potencjału pola $\Phi$.
    \item \textbf{Unifikacja wszystkich czterech oddziaływań fundamentalnych} poprzez ich dynamiczne zależności od pola $\Phi$.
    \item \textbf{Nowa klasyfikacja i model stabilności czarnych dziur}, przewidujący maksymalną masę (rzędu $10^6 M_\odot$) oraz czasy życia metastabilnych czarnych dziur gwiazdowych poprzez kwantowe tunelowanie.
    \item \textbf{Nowa wizja kosmologii cyklicznej}, gdzie Wszechświat przechodzi przez nieskończone cykle Wielkich Wybuchów, inicjowanych lokalnie w regionach powracających do stanu maksymalnej "czystości" przestrzeni.
\end{itemize}

mTGP generuje \textbf{konkretne i testowalne przewidywania} w dziedzinie fizyki kwantowej (np. modyfikacje zasady nieoznaczoności), astrofizyki (np. sygnatury w falach grawitacyjnych, odchylenia w kosmologicznych pomiarach odległości, modyfikacje obfitości pierwiastków z BBN, zmienne stosunki mas cząstek), a także oferuje alternatywne wyjaśnienia dla obserwowanych anomalii w rotacji galaktyk.

Mimo że mTGP stawia przed nami szereg wyzwań, takich jak pełne kwantyzowanie pola $\Phi$ czy szczegółowe modelowanie zachowania energii i pędu w dynamicznej czasoprzestrzeni, jej spójność wewnętrzna i zdolność do oferowania rozwiązań dla fundamentalnych problemów czynią ją \textbf{obiecującym kierunkiem dalszych badań}. Teoria Generowanej Przestrzeni stanowi odważny krok w kierunku zunifikowanego opisu Wszechświata, gdzie przestrzeń i materia są nierozerwalnie ze sobą związane w dynamicznym tańcu tworzenia.

\begin{thebibliography}{99}
\bibitem{Serafin2025} M. Serafin, "Teoria Generowanej Przestrzeni: W kierunku Unifikacji Grawitacji i Mechaniki Kwantowej," viXra:2505.0171 (2025). 
\bibitem{Webb2011} J. K. Webb et al., "Further evidence for cosmological evolution of the fine structure constant," Phys. Rev. Lett. \textbf{107}, 191101 (2011) [\url{arXiv:1008.3907}]. 
\bibitem{Riess2018} A. G. Riess et al., "Large Magellanic Cloud Cepheid Standards Provide a 1.9\% Distance Ladder Measurement of the Hubble Constant," Astrophys. J. \textbf{855}(2), 136 (2018) [\url{arXiv:1804.10655}]. 
\bibitem{Planck2018} Planck Collaboration, "Planck 2018 results. VI. Cosmological parameters," Astron. Astrophys. \textbf{641}, A6 (2020) [\url{arXiv:1807.06209}]. 

% References imported from main.tex and unification.tex (original context)
\bibitem{Einstein1915} A. Einstein, "Die Feldgleichungen der Gravitation," \textit{Sitzungsberichte der Preussischen Akademie der Wissenschaften zu Berlin}, 844-847 (1915). 
\bibitem{Schrodinger1926} E. Schr\"odinger, "An Undulatory Theory of the Mechanics of Atoms and Molecules," \textit{Physical Review} \textbf{28}(6), 1049-1070 (1926). 
\bibitem{Heisenberg1927} W. Heisenberg, "Ueber den anschaulichen Inhalt der quantentheoretischen Kinematik und Mechanik," \textit{Zeitschrift f\"ur Physik} \textbf{43}(3-4), 172-198 (1927). 
\bibitem{Dirac1928} P. A. M. Dirac, "The Quantum Theory of the Electron," \textit{Proceedings of the Royal Society of London. Series A, Containing Papers of a Mathematical and Physical Character} \textbf{117}(778), 610-624 (1928). 
\bibitem{MTW} C. W. Misner, K. S. Thorne, J. A. Wheeler, \textit{Gravitation}, W. H. Freeman (1973). 
\bibitem{Wald1984} R. M. Wald, \textit{General Relativity}, University of Chicago Press (1984). 
\bibitem{BransDicke1961} C. Brans and R. H. Dicke, "Mach's Principle and a Relativistic Theory of Gravitation," \textit{Physical Review} \textbf{124}(3), 925-935 (1961). 
\bibitem{Moffat2004} J. W. Moffat, "A New Theory of Gravity," \textit{Phys. Rev. D} \textbf{70}, 124009 (2004) [\url{arXiv:gr-qc/0010078}]. 
\bibitem{Magueijo2003} J. Magueijo, "Covariant theories of varying c," \textit{Rept. Prog. Phys.} \textbf{66}, 2025-2068 (2003) [\url{arXiv:astro-ph/0305457}]. 
\bibitem{Perlmutter1999} S. Perlmutter et al. (Supernova Cosmology Project), "Measurements of Omega and Lambda from 42 High-Redshift Supernovae," \textit{Astrophysical Journal} \textbf{517}(2), 565-586 (1999) [\url{arXiv:astro-ph/9812133}]. 
\bibitem{Riess1998} A. G. Riess et al. (Supernova Search Team), "Observational Evidence from Supernovae for an Accelerating Universe and a Cosmological Constant," \textit{Astronomical Journal} \textbf{116}(3), 1009-1038 (1998) [\url{arXiv:astro-ph/9805201}]. 
\bibitem{Weinberg1979} S. Weinberg, "Ultraviolet divergences in quantum theories of gravitation," in \textit{General Relativity: An Einstein Centenary Survey}, ed. S.W. Hawking and W. Israel, Cambridge University Press, 790-831 (1979). 
\bibitem{Verde2019} L. Verde, T. Treu, A. G. Riess, "Tensions between the Early and the Late Universe," \textit{Nature Astronomy} \textbf{3}, 891-895 (2019) [\url{arXiv:1907.10625}]. 
\bibitem{WeinbergQFT1} S. Weinberg, \textit{The Quantum Theory of Fields, Vol. 1: Foundations}, Cambridge University Press (1995). 
\bibitem{GriffithsQM} D. J. Griffiths, \textit{Introduction to Quantum Mechanics}, Pearson Prentice Hall (2005). 
\bibitem{CarrollGR} S. M. Carroll, \textit{Spacetime and Geometry: An Introduction to General Relativity}, Cambridge University Press (2004). 
\bibitem{PeskinSchroeder} M. E. Peskin, D. V. Schroeder, \textit{An Introduction To Quantum Field Theory}, Westview Press (1995). 
\bibitem{KolbTurner} E. W. Kolb, M. S. Turner, \textit{The Early Universe}, Westview Press (1990). 
\bibitem{MukhanovCosmo} V. Mukhanov, \textit{Physical Foundations of Cosmology}, Cambridge University Press (2005). 
\bibitem{Higgs1964} P. W. Higgs, "Broken Symmetries and the Masses of Gauge Bosons," \textit{Phys. Rev. Lett.} \textbf{13}(16), 508-509 (1964). 
\bibitem{EnglertBrout1964} F. Englert, R. Brout, "Broken Symmetry and the Mass of Gauge Vector Mesons," \textit{Phys. Rev. Lett.} \textbf{13}(9), 321-323 (1964). 
\bibitem{Guralnik1964} G. S. Guralnik, C. R. Hagen, T. W. B. Kibble, "Global Conservation Laws and Massless Particles," \textit{Phys. Rev. Lett.} \textbf{13}(20), 585-587 (1964). 
\bibitem{ATLASHiggs} G. Aad et al. (ATLAS Collaboration), "Observation of a new particle in the search for the Standard Model Higgs boson with the ATLAS detector at the LHC," \textit{Phys. Lett. B} \textbf{716}(1), 1-29 (2012) [\url{arXiv:1207.7214}]. 
\bibitem{CMSHiggs} S. Chatrchyan et al. (CMS Collaboration), "Observation of a new boson with mass near 125 GeV in pp collisions at $\sqrt{s}=7$ and 8 TeV," \textit{Phys. Lett. B} \textbf{716}(1), 30-61 (2012) [\url{arXiv:1207.7235}]. 
\bibitem{Sakharov1967} A. D. Sakharov, "Vacuum Quantum Fluctuations in Curved Space and the Theory of Gravitation," \textit{Dokl. Akad. Nauk SSSR} \textbf{177}, 70-71 (1967) [Sov. Phys. Dokl. \textbf{12}, 1040-1041 (1968)]. 
\bibitem{Verlinde2011} E. P. Verlinde, "On the Origin of Gravity and the Laws of Newton," \textit{JHEP} \textbf{1104}, 029 (2011) [\url{arXiv:1001.0785}]. 
\bibitem{Milgrom1983} M. Milgrom, "A modification of the Newtonian dynamics as a possible alternative to the hidden mass hypothesis," \textit{Astrophys. J.} \textbf{270}, 365-370 (1983). 
\bibitem{Rovelli2004} C. Rovelli, \textit{Quantum Gravity}, Cambridge University Press (2004). 
\bibitem{Polchinski1998} J. Polchinski, \textit{String Theory Vol. 1: An Introduction to the Bosonic String}, Cambridge University Press (1998). 
\end{thebibliography}

\end{document}